\chapter{\label{ch:5-bulk}Bulk RNA-seq analysis of PRS inhibitors}

%\minitoc

\section{Introduction}
Although MM treatment has improved significantly in the last 10-20 years, MM remains an incurable disease.
Most MM patients relapse and become resistant to drugs they have previously been treated with.
Therefore, research into novel therapeutics that can overcome multi-drug resistance and can be used to treat relapsed patients is of great importance.

There is a new exciting class of compounds for the treatment of many diseases, including numerous cancers.
This class of compounds are analogues of, and have been derived from the drug Febrifugine.
Febrifugine is the biologically active component of the herb \textit{Dichroa febrifuga}, which is considered one of the fundamental herbs in traditional Chinese medicine.
Febrifugine was shown to have strong anti-malarial effects.
One such Febrifugine derivative, Halofuginone (HF), has been shown to inhibit T Helper 17 (TH17) cell differentiation, by activating the amino acid response (AAR)\cite{sundrud2009halofuginone}.
Halofuginone inhibits the enzyme glutamyl-prolyl tRNA synthetase (EPRS).
EPRS is a bifunctional aminoacyl-tRNA synthetase (AARS) and catalyses the the aminoacylation of glutamic acid and proline tRNA species (such that it charges its cognate tRNAs with glutamic acid and proline).
Halofuginone and Febrifugine compete with proline at the prolyl-tRNA synthetase active site of EPRS, specifically targeting utilisation of proline during translation\cite{keller2012halofuginone}.
This results in an accumulation of uncharged tRNA\textsuperscript{pro}s, giving the same cellular environment as if the cell were proline deficient, triggering the AAR to respond to the apparent proline deprived state.

AARSs are essential in protein synthesis, aiding in building chains of amino acids.
Human cancer cells often have an increased rate of protein synthesis, this is especially true in multiple myeloma, creating huge amounts of non-functional paraprotein, therefore are more reliant on aaRSs.
As discussed in chapter \ref{ch:2-litreview}, HF has previously shown anti-MM activity in vitro and in-vivo.
HF induced cytotoxicity and apoptosis in numerous MM cell-lines and primary MM cells.
HF was also shown to inhibit MM growth and prolong survival in a mouse xenograft MM model.
However, the mechanism by which HF exerted its affect was not elucidated, and it is not clear if the AAR plays a role in HF's effectiveness in MM.

It has also been shown that HF's anti-MM effect can be reduced in the presence of excessive proline.
Tumours have more proline than healthy cells <REF>.
This means that HF has a very narrow therapeutic window, and exhibits many side effects.

% Ralph drug drawing
\begin{figure}[ht]
    \centering
    \includegraphics[width=0.9\textwidth]{figures/Results/Transcriptomics/ralph_figure.pdf}
    \caption[Halofuginone and NCP26 structures]{Diagrams of Halofuginone/MAZ1392 (a) and NCP26 (b) and their chemical structures.
    Halofuginone is an ATP dependent, proline and tRNA competitive ProRS inhibitor.
    NCP26 is an ATP competitive and proline uncompetitive ProRS inhibitor.
    Aminoacylation is an ATP-dependent process, requiring ATP to activate amino acids.
    Figures by Ralph Mazitschek.
    }
    \label{fig:ralph_diagrams}
\end{figure}

Recently, The Mazitschek group have synthesized numerous other compounds which target the ProRS site of EPRS.
One such example, NCP26 (figure \ref{fig:ralph_diagrams}), binds to the ATP binding site of ProRS, inhibiting utilization of ATP.
NCP26 does not compete with proline for the active site of ProRS, unlike Febrifugine and Halofuginone.
Aminoacylation is an ATP-dependent process so blocking ATP binding inhibits this process, and also leads to an accumulation of uncharged tRNA\textsuperscript{pro}s.
NCP26 will hopefully alleviate the issues associated with HF treatment.
More PRS inhibitors have been synthesized by the group, including NCP22.

% Flush figures
\clearpage

\section{Cell-based assay results}

%%% Carf dose response curve to check sensitive and resistant to carf

\subsection{Halofuginone and NCP26 are cytotoxic to drug sensitive and drug resistant MM cell lines in a dose-dependent manner}
The effect of the PRS inhibitors NCP22, NCP26, MAZ1805 (Halofuginol) and MAZ1392 (Halofuginone; HF) on cell viability was investigated using the MM cell lines AMO-1 and L363 CFZ-r.
The cell lines were treated with a range of concentrations of each compound (3.9\si{\nano\Molar}- 1\si{\micro\Molar}).
Cell viability was assessed using presto-blue assays (section \ref{subsec:method_doseresponse}), and dose response curves were generated (Figures \ref{fig:dose}a and \ref{fig:dose}c).
Halofuginone and NCP26 reduced cell viability of PI-sensitive AMO-1 cells and carfilzomib resistant L363 cells in a dose-dependent fashion.
For this concentration range, MAZ1805 and NCP22 seemed to have little effect on cell viability of WT or CFZ-r cells, and IC\textsubscript{50} values were unable to be calculated.
Halofuginone was found to be more potent than NCP26.

% dose response figure
\begin{figure}[h]
\centering
\includegraphics[width=0.9\textwidth]{figures/Results/dose_response/MAZ_ncp_wt_cfz.pdf}
\caption[PRS inhibitor dose response curves]{Multiple myeloma (MM) cell lines treated with PRS inhibitors- MAZ1392 (Halofuginone), MAZ1805, NCP26 and NCP22.
MM cell lines were treated for 48 hours with a range of concentrations (3.91\si{\nano\Molar}-1\si{\micro\Molar}) of PRS inhibitors.
a) and c) Dose response curves.
a and b) WT AMO-1 cells.
c) and d) Carfilzomib resistant L363 cells.
b) and d) Proportion of cells viable after 48 hours of 1\si{\micro\Molar} treatment with each agent.}
\label{fig:dose}
\end{figure}
%%

For WT AMO-1 cells, Halofugione had an IC\textsubscript{50} of 141.8nM and NCP26 an IC\textsubscript{50} of 502nM. For CFZ-r cells, Halofuginone had an IC50 of 1185nM and NCP26's IC\textsubscript{50} was ambiguous, a higher stock concentration would be required for calculation.
Figure \ref{fig:dose}b and \ref{fig:dose}d show the proportion of viable cells following 48 hours of treatment of the PRS inhibitors.
WT AMO-1 cells were found to be more sensitive to NCP26 and Halofuginone treatment than carfilzomib resistant L363 cells.
This may indicate some acquired cross-resistance built up from carfilzomib exposure or inherent resistance in the L363 cell line over the AMO-1 cell line.

\subsection{Carfilzomib and NCP26 have an additive or mild antagonistic effect together}\label{subsec:synergy}
Drug combinations have proved effective in MM in recent years, for example the combination of bortezomib, lenalidomide, and dexamethasone (VRd) is used extensively for newly diagnosed MM patients.
Drugs are often used in combination so that outcomes are improved (synergistic efficacy) or to reduce off-target effects and toxicity by minimizing the doses of the drugs (synergistic potency) \cite{meyer2019quantifying}.

To assess if NCP26 and carfilzomib work together synergistically, AMO-1 cells were treated with varying concentrations of NCP26 and Carfilzomib for 72 hours, then presto blue assays were performed to determine cell viability (see figure \ref{fig:synergy}).
SynergyFinder \cite{zheng2021synergyfinder} was used to calculate the compounds' synergy scores (-4.66 ZIP; -4.18 Loewe; -5.53 Bliss).
From these values it is unlikely that NCP26 and Carfilzomib work together synergistically.
NCP26 and carfilzomib seem to have an additive effect together, or slight antagonistic effect.
This reflects a previous result where HF demonstrated moderate antagonism with the bortezomib \cite{leiba2012halofuginone}.

% synergy figure
\begin{figure}[h]
\centering
\includegraphics[width=0.9\textwidth]{figures/Results/Transcriptomics/synergy.pdf}
\caption[NCP26 and carfilzomib synergism]{Investigating potential synergy between NCP26 and carfilzomib.
AMO-1 cells were treated with varying concentration combinations of NCP26 and carfilzomib for 72 hours, then cell viability was determined using presto blue assays.
\textbf{A)} Dose response curves for NCP26 with different carfilzomib concentrations.
\textbf{B)} Matrix view of NCP26 and Carfilzomib concentration responses.
}
\label{fig:synergy}\end{figure}
%%

% Flush figures
\clearpage

\section{Bulk RNA-seq}

\subsection{Experiment overview}
The transcriptome of drug-sensitive and drug-resistant MM cells treated with four ProRS inhibitors was investigated using bulk RNA-seq.
PI-sensitive WT AMO-1 cells and carfilzomib resistant L363 cells (CFZr) were used.
Cells were treated for 6 and 24 hours with a DMSO control, or 1\si{\micro\Molar} of a ProRS inhibitor (MAZ1392 (Halofuginone), NCP26, NCP22 and MAZ1805 (Halofuginol)).
WT cells were also treated with 100\si{\nano\Molar} carfilzomib.
CFZr cells were treated in the presence of 100\si{\nano\Molar} carfilzomib.
The computational workflow for bulk RNA-seq analysis is outlined in section ...

\subsection{Clustering}

% bulk clustering figure
\begin{figure}[p]
\centering
\includegraphics[width=0.9\textwidth]{figures/Results/Transcriptomics/bulk_clustering_maz_ncp_wt_cfz.pdf}
\caption[Bulk RNA-seq sample clustering]{Bulk RNA-seq sample clustering.
WT AMO-1 cells and CFZr L363 cells treated for 6 or 24 hours with a DMSO control, 100nM carfilzomib or 1uM of a PRS inhibitor (MAZ1392/Halofuginone, MAZ1805/Halofuginol, NCP26 and NCP22).
Clustering analysis of sample-sample distances (a, c and e) and principal component analysis (PCA; b, d and f).
a) and b) both cell types (WT and CFZr) displayed; c) and d) WT AMO-1 only; e) and f) CFZr samples only.
}
\label{fig:clustering_bulk}\end{figure}
%%

Figure \ref{fig:clustering_bulk} shows clustering analysis of the samples.
Figure \ref{fig:clustering_bulk} A and B show the samples distinctly separate into respective cell types, and this makes up the majority (69\%) of the variance in the dataset.
Therefore, the different cell types (WT and CFZr) were separated and analysed individually.

The less active compounds from IC\textsubscript{50} data, MAZ1805 and NCP22, cluster closely with the DMSO-treated controls.
The more active inhibitors, NCP26 and Halofuginone, cluster separately from DMSO controls and less active inhibitors, indicating they have elicited a stronger transcriptional response.
At 6 hours NCP26 and Halofuginone cluster closely together.
At 24 hours NCP26 and Halofuginone separate more.
This may suggest that they work slightly differently in the cells, or this could just be reflecting the differences in their potency.

In WT cells, carfilzomib clusters separately from DMSO and the less active inhibitors.
At 6 hours, carfilzomib is separate from the NCP26 and Halofuginone cluster, but clusters more closely at 24 hours.
This could suggest an initial difference in mechanism of action and transcriptional response to the ProRS inhibitors, but culminating in a similar response as time progresses, such as cell stress and cell death pathways.

% PCA pathway figure
\begin{figure}[htb]
\centering
\includegraphics[width=0.9\textwidth]{figures/Results/Transcriptomics/bulk_clustering_PCs_analysed.pdf}
\caption[PCA pathway enrichment]{PCA pathway enrichment.
Pathway enrichment analysis performed using REACTOME of top contributing genes from principal component analysis (PCA).
A) WT AMO-1 dataset with carfilzomib-treated samples removed.
PC1 seems to account for the separation between DMSO controls/ less active compounds and the more active ProRS inhibitors (NCP26 and Halofuginone).
Genes contributing positively towards PC1 are upregulated in NCP26/ Halofuginone compared to controls/ less active compounds.
Enriched pathways from top genes in PC1 shown beneath PCA plot.
B) CFZr L363 dataset.
PC2 seems to account for the separation between DMSO controls/ less active compounds and the more active ProRS inhibitors (NCP26 and Halofuginone).
Genes contributing negatively towards PC2 (down arrow), are upregulated in NCP26/ Halofuginone compared to controls.
Pathways enriched from top genes in PC2 shown beneath PCA plot.}
\label{fig:pca_pathway}
\end{figure}
%%

The principal components of the WT AMO-1 and CFZr datasets were examined more closely.
Carfilzomib-treated samples were removed from the WT dataset, and differential expression, variance stabilising transformation and PCA was performed again.
This was to ensure that the difference between controls and active ProRS inhibitors was captured in PC1 or PC2 following dimensionality reduction.
Pathway enrichment analysis was performed for the top genes from the principal components accounting for the difference between controls and Halofuginone/NCP26 treatment (PC1 for WT cells, PC2 for CFZr cells; Figure \ref{fig:pca_pathway}).
The pathways `ATF4 activates genes in response to ER stress', `Response of GCN2 to amino acid deficiency' and `cytosolic tRNA aminoacylation' were all enriched, suggesting that the amino acid response is activated.
Additionally, genes involved in the cell cycle and G1/S transition contributed to this treatment-related separation.

\subsection{Drug sensitive MM}

\subsubsection{Differential expression}
For WT cells at 6 hours, 2119 genes were differentially expressed ($\lvert log_{2}FC \rvert$ > 0.5 and p\textsubscript{adj}<0.05; DE) for NCP26 treated-samples, 3019 DE genes (DEGs) for Halofuginone, 33 DE genes for MAZ1805, 218 DE genes for NCP22, and 983 DE genes for carfilzomib-treated samples compared to DMSO controls.
At 24 hours, NCP26-treated samples had 3323 DE genes, Halofuginone 3426, MAZ1805 2, NCP22 1, and carfilzomib 2260 DE genes compared to DMSO controls.
DEGs for NCP26 and Halofuginone treatment are shown in figure \ref{fig:wt_de}.
%
% DE volcano scatter WT %
\begin{figure}[htb]
\centering
\includegraphics[width=0.9\textwidth]{figures/Results/Transcriptomics/WT/wt_de_volcano_scatter.pdf}
\caption[Differentially expressed genes WT AMO-1 cells]{Differentially expressed genes (DEGs) for Halofuginone and NCP26 treated WT AMO-1 cells at 6 and 24 hours.
Scatter plot of genes for 6 hours treated vs 24 hours treated.
Red points indicate genes which are differentially expressed (p\textsubscript{adj} < 0.01) at both 6 and 24 hours.
Volcano plots are also shown.
Blue points indicate downregulated DEGs (p\textsubscript{adj} < 0.01 \& $\log_{2}FC < -1$).
Red points indicate upregulated DEGs (p\textsubscript{adj} < 0.01 \& $\log_{2}FC > 1$).
Top DEGs are annotated with HGNC symbols.
}
\label{fig:wt_de}
\end{figure}
%%
Genes highly differentially expressed by the ProRS inhibitors are coloured and annotated with their gene symbol.
Numerous genes involved in stress response pathways can be seen to be upregulated following NCP26 and Halofuginone treatment, including \textit{TRIB3}, \textit{JUN} and \textit{ATF3}.
Additionally, various histone genes are upregulated.
Some genes involved in cell cycle progression, such as \textit{CDKN1A}, are downregulated following ProRS inhibition.

The transcriptional changes for 6 hours of exposure to NCP26 were correlated with proteomic changes for the same treatment condition (figure \ref{fig:proteomic_rna_scatter}).
The proteomic data was supplied by collaborators [COLLABORATORs names here].
% scatter plot proteomics vs rna-seq %
\begin{figure}[htb]
\centering
\includegraphics[width=0.5\textwidth]{figures/Results/Transcriptomics/WT/rna_vs_proteomics_ncp26_6h_scatter.pdf}
\caption[]{Scatterplot of proteomic and RNA-seq datasets depicting changes after 6 hr NCP26 exposure to AMO-1 cells.
Red points indicate genes which are differentially expressed (p\textsubscript{adj} < 0.01) in both RNA-seq and proteomic datasets.
Folllowing NCP26 treatment, TRIB3 and INHBE (both \textit{ATF4} targets) were the proteins with the highest increase in abundance ($\log_{2}$FC = 0.64 and INHBE  $\log_{2}$FC = 0.58) in the proteomic dataset and both were significantly upregulated in the RNA-seq dataset.
}
\label{fig:proteomic_rna_scatter}
\end{figure}
%%
%
Consistent with a mechanism of a global reduction of protein synthesis (with the exception of preferential translation of \textit{ATF4}) upon  ISR induction and eIF2$\alpha$ phosphorylation, very few proteins with an increased abundance were identified (52 proteins with a $\log_{2}$ fold-change between 0.2 and 0.64).
Additionally, a larger shift of proteins with lower abundance in NCP26 treated samples compared to DMSO controls was found.
This correlates with transcriptional data where more DEGs were downregulated than upregulated for all ProRS treatment conditions compared to DMSO controls.
Also, in-fitting with this mechanism, selective \textit{ATF4} target genes are upregulated, such as \textit{TRIB3} and \textit{INHBE} (as seen highlighted in figure \ref{fig:proteomic_rna_scatter}).
This shows a dominant role of the integrated stress response at 6 hours on the transcriptomic and proteomic level.


\subsubsection{Pathway enrichment analysis}
Pathway enrichment analysis was performed for the top DEGs for NCP26 and Halofuginone treatment.
REACTOME and Gene ontology biological processes (GOBP) pathways were explored.
FIGURE OF PATHWAY analyses.


% AAR heatmap
\begin{figure}[p]
\centering
\includegraphics[width=0.5\textwidth]{figures/Results/Transcriptomics/WT/wt_halo_ncp26_aar_heatmap.pdf}
\caption[Amino acid starvation response genes heatmap WT cells]{Amino acid starvation response (AAR) genes heatmap for WT cells.
Differentially expressed genes (DEGs) from WT AMO-1 cells intersected with genes involved in the AAR.
A list of known AAR genes was compiled by collating AAR genesets from the Molecular Signatures Database (MSigDb).
The colour scale shows $\log_{2}$ fold change of expression for each treated sample, compared to its DMSO time control.
Red indicates an upregulated gene and blue a downregulated gene.
}
\label{fig:wt_aar_heatmap}
\end{figure}
%%

The effects of NCP26 and Halofuginone treatment on the amino acid starvation response was examined in more detail.
AAR genesets `GOBP response to amino acid starvation', `REACTOME response of EIF2AK4/GCN2 to amino acid deficiency' and `KRIGE amino acid deprivation' were collated from the Molecular Signatures Database (MSigDb), making up a list of 166 unique genes.
This list of AAR genes was intersected with genes that were differentially expressed for any of the ProRS inhibitor-treated samples (i.e NCP26 vs DMSO and Halofuginone vs DMSO at 6 or 24 hours).
Figure \ref{fig:wt_aar_heatmap} shows a heatmap of DE AAR genes for WT AMO-1 cells.
The AAR transcription factors \textit{ATF3}, \textit{DDIT3} (CHOP), \textit{CEBPB} and \textit{CEBPG} are all markedly upregulated following NCP26 and Halofuginone treatment.
Genes coding for aminoacyl tRNA synthetases are also shown to be upregulated, such as \textit{WARS1}, \textit{SARS1} and \textit{CARS1}.
Amino acid transporters, such as \textit{SLC7A11}, were also upregulated following ProRS treatment.

A list of 287 unique genes activated by \textit{ATF4} was compiled from the genesets: `REACTOME ATF4 activates genes in response to endoplasmic reticulum stress' and `ATF4 Q2' (genes having at least one occurrence of the transcription factor binding site V.ATF4 Q2 in the regions spanning up to 4 kb around their transcription starting sites).
Figure \ref{fig:wt_atf4_heatmap} shows a heatmap for ATF4 activated genes and genes DE by NCP26 and Halofuginone.
Numerous \textit{ATF4} targets can be seen to be differentially expressed following ProRS inhibition with NCP26 and Halofuginone.
Of the 287 genes in the list, 165 were differentially expressed ($\lvert log_{2} \rvert$FC > 0.5 and p\textsubscript{adj}<0.05) following NCP26 or Halofuginone treatment.


% ATF4 heatmap
\begin{figure}[p]
\centering
\includegraphics[width=0.5\textwidth]{figures/Results/Transcriptomics/WT/wt_halo_ncp26_atf4_heatmap.pdf}
\caption[Heatmap of \textit{ATF4} activated genes for ProRS treated WT cells]{Heatmap of \textit{ATF4}activated genes for WT cells treated with NCP26 and Halofuginone.
Differentially expressed genes (DEGs) from WT AMO-1 cells intersected with \textit{ATF4} activated genes.
A list of genes activated by the transcription factor \textit{ATF4} was compiled by collating genesets from the Molecular Signatures Database (MSigDb).
The colour scale shows $\log_{2}$ fold change of expression for each treated sample, compared to its DMSO time control.
Red indicates an upregulated gene and blue a downregulated gene.
}
\label{fig:wt_atf4_heatmap}
\end{figure}
%%

Taken together it is clear that the amino acid starvation response is activated in MM cells by NCP26 and HF treatment.

%% vs carfilzomib
\subsubsection{ProRS inhibitors vs carfilzomib}
The mechanism of action of proteome inhibitors has been extensively studied and well described.
Figure \ref{fig:clustering_bulk}d demonstrates similarities at 24 hours between the transcriptional effects of ProRS inhibitors and the proteasome inhibitor carfilzomib on AMO-1 cells.
It also demonstrates ProRS and carfilzomib-treated samples separation at 6 hours, highlighting differences in their initial mechanism of action.

The similarities and differences between ProRS inhibitors and carfilzomib were studied in more detail.
As seen by x-y trends of scatter plots and number of overlapping genes in venn diagrams (figure \ref{fig:wt_carf_compare}a and b), ProRS inhibitors and carfilzomib treatment are more similar at 24 hours, and share more DEGs (compared to DMSO controls), than at 6 hours.


Carfilzomib treatment resulted in a pronounced induction of the heat shock response, changes to ubiquitin mediated processes and protein folding, in line with the well-defined cellular changes as a result of proteasome inhibition.
At 6 hours, some ISR effectors, such as \textit{ATF3} and \textit{JUN}, are upregulated by carfilzomib and NCP26/Halofuginone treatment.
%Suggesting that ProRS inhibitors may have a distinct enough mechanism of action to bypass proteasome resistance.

At 24 hours, CFZ and ProRS inhibitor treatment seem to culminate in similar end-stage stress, cell cycle changes and apoptotic mechanisms.
Both ProRS inhibition and CFZ treatment result in pathway enrichment of `cell cycle arrest', `negative regulation of cell proliferation', `positive regulation of apoptosis process' and `type 1 interferon signalling pathway'.
Figure \ref{fig:wt_carf_compare}d shows a heatmap of NCP26, Halofuginone and CFZ for $\log_{2}FC$ compared to DMSO controls at 6 and 24 hours for selected genes, belonging to various pathways/ classes.
NCP26 and Halofuginone demonstrate similar effects on myeloma markers as CFZ, such as downregulating the MM pathological marker \textit(SDC1/ CD138).
CFZ is an established anti-MM therapy approved in the clinic, therefore this is promising for the effectiveness of ProRS inhibitors in MM.



% carf vs proRS heatmap
\begin{figure}[p]
\centering
\includegraphics[width=\textwidth]{figures/Results/Transcriptomics/WT/carf_halo_ncp_comparison_figure.pdf}
\caption[ProRS inhibitors compared with Carfilzomib's mechanism of action]{ProRS inhibitors compared with Carfilzomib's mechanism of action.
a) Venn diagrams showing overlapping differentially expressed genes (DEGs; upregulated or downregulated following treatment)at 6 and 24 hours.
b) Scatter plots showing correlation of carfilzomib DEGs against Halofuginone or NCP26 DEGs.
c) Pathway analysis (Gene onotology biological processes; GOBP) for top upregulated genes.
d) Heatmap of selected differentially expressed genes upon carfilzomib, Halofuginone and NCP26 treatment.
}
\label{fig:wt_carf_compare}
\end{figure}
%%

%%%%%%%%%%%%%%%%%%%%%%%%%%%%%%%%%%%%%%%%%%%%%%%%%%%%%%%%%%%%%%%%%%%%%%%%%%%%%%%%%%%%%%%%%%%%%%%%%%%%%%%%%%%%%%%%%%%%%%%%
%%%%%%% CFZ L363 %%%%%%%%%
\clearpage
\subsection{Carfilzomib resistant cells}

\subsubsection{Differential expression}
For CFZr L363 cells at 6 hours, 1165 genes were DE for NCP26 treated-samples, 2424 DE genes (DEGs) for Halofuginone, 222 DE genes for MAZ1805, and 0 DE genes for NCP22-treated samples compared to DMSO controls.
At 24 hours, NCP26-treated samples had 852 DE genes, Halofuginone 256 DE, and MAZ1805 and NCP22 no DE genes compared to DMSO controls.
DEGs for NCP26 and Halofuginone treatment are shown in figure \ref{fig:cfz_de}.
% DE volcano scatter WT %
\begin{figure}[htb]
\centering
\includegraphics[width=0.9\textwidth]{figures/Results/Transcriptomics/CFzr/cfz_de_volcano_scatter.pdf}
\caption[Differentially expressed genes CFZr L363 cells]{Differentially expressed genes (DEGs) for Halofuginone and NCP26 treated CFZr L363 cells at 6 and 24 hours.
Scatter plot of genes for 6 hours treated vs 24 hours treated.
Red points indicate genes which are differentially expressed (p\textsubscript{adj} < 0.01) at both 6 and 24 hours.
Volcano plots are also shown.
Blue points indicate downregulated DEGs (p\textsubscript{adj} < 0.01 \& $\log_{2}FC < -1$).
Red points indicate upregulated DEGs (p\textsubscript{adj} < 0.01 \& $\log_{2}FC > 1$).
Top DEGs are annotated with HGNC symbols.
}
\label{fig:cfz_de}
\end{figure}
%%
Numerous genes involved in the AAR can be seen to be upregulated following NCP26 and Halofuginone treatment, including \textit{TRIB3}, \textit{ATF3}, \textit{CHAC1}, \textit{INHBE}, \textit{PSAT1}, \textit{ASNS} and \textit{SESN2}.
As well as amino acid transporters \textit{SLC7A11} and \textit{SLC6A9}, and tRNA aminoacyl synthetase \textit{WARS1}.

\subsubsection{Pathway enrichment analysis}
Pathway enrichment analysis was performed for the top DE genes for NCP26 and Halofuginone treated samples compared to DMSO controls (figure \ref{fig:cfz_pathway}).
Multiple pathways relating to endoplasmic reticulum stress and apoptosis were enriched following ProRS inhibition.
`Cellular repsonse to amino acid starvation' was enriched, along with `amino acid transport', indicating that NCP26/Halofuginone likely activate the amino acid starvation response in the CFZ resistant MM cells.
Additionally the pathway `response to unfolded protein' was enriched.
The unfolded protein response is part of the integrated stress response so shares many of the same effectors as the AAR, such as \textit{DDIT3/CHOP} overexpression.

% Pathway analysis %
\begin{figure}[htb]
\centering
\includegraphics[width=0.6\textwidth]{figures/Results/Transcriptomics/CFzr/CFZ_pathway_analysis.eps}
\caption[Pathway analysis for ProRS inhibitor-treated CFZr cells]{Pathway analysis for ProRS treated CFZr cells at 24 hours [CHECK].
Gene ontology biological processes (GOBP) was performed for the top DE genes.
}
\label{fig:cfz_pathway}
\end{figure}
%
The amino acid response was investigated more closely for the PI-resistant cell line.
Figure \ref{fig:cfz_aar_heatmap} shows a heatmap of differentially expressed genes for NCP26/Halofuginone treated CFZr cells at 6 or 24 hours, which are known members of the AAR (collated from MSigDbl; as seen above).

% AAR heatmap
\begin{figure}[p]
\centering
\includegraphics[width=0.5\textwidth]{figures/Results/Transcriptomics/CFzr/cfz_halo_ncp26_aar_heatmap.pdf}
\caption[Amino acid starvation response genes heatmap CFZr cells]{Amino acid starvation response (AAR) genes heatmap for CFZr cells.
Differentially expressed genes (DEGs) from CFZr AMO-1 cells intersected with genes involved in the AAR.
A list of known AAR genes was compiled by collating AAR genesets from the Molecular Signatures Database (MSigDb).
}
\label{fig:cfz_aar_heatmap}
\end{figure}
%%
A strong elicitation of the AAR can be seen following NCP26 and Halofuginone treatment.
The transcription factors \textit{CEBPG}, \textit{CEBPB}, \textit{ATF3}, \textit{DDIT3} are strongly upregulated, along with amino acid transporter \textit{SLC7A11} and tRNA aminoacyl synthetase \textit{WARS1}. 

% ATF4 heatmap
\begin{figure}[p]
\centering
\includegraphics[width=0.5\textwidth]{figures/Results/Transcriptomics/CFzr/cfz_halo_ncp26_atf4_heatmap.pdf}
\caption[Heatmap of \textit{ATF4} activated genes for ProRS treated CFZr cells]{Heatmap of \textit{ATF4} activated genes for CFZr L363 cells treated with NCP26 and Halofuginone.
\textit{ATF4} activated genes, which are differentially expressed following NCP26 or Halofuginone treatment at 6 or 24 hours.
A list of genes activated by the transcription factor \textit{ATF4} was compiled by collating genesets from the Molecular Signatures Database (MSigDb).
The colour scale shows $\log_{2}$ fold change of expression for each treated sample, compared to the corresponding DMSO control.
Red indicates an upregulated gene and blue a downregulated gene.
}
\label{fig:cfz_atf4_heatmap}
\end{figure}
%%

CFZ-resistant L363 cells show a similar engagement of the AAR and apoptotic mechanisms following NCP26 and Halofuginone treatment.
Thus targeting the ProRS could be a potential effective strategy in overcoming PI drug resistance in MM\@.

% flush figures
\clearpage
\section{Summary}
NCP26 and Halofuginone have been shown to reduce cell viability of drug sensitive and drug resistant MM cell lines in a dose-dependent manner.

A general shift towards lower protein abundance and downregulation of gene expression has been shown to occur following exposure of MM cells to NCP26, indicating activation of components of the integrated stress response.
However, \textit{ATF4} target genes were shown to be upregulated and larger in abundance following NCP26 and Halofuginone treatment.
\textit{ATF4} is the master regulator of amino acid metabolism.
It is activated by amino acid deprivation.
Halofuginone and NCP26 treatment caused increased expression of \textit{ATF4} in both drug sensitive and PI-resistant MM cell lines.
Expression of numerous genes involved in the amino acid starvation and integrated stress response were markedly increased following Halofuginone and NCP26 treatment.
\textit{DDIT3} and downstream pro-apoptotic genes' expression was increased following NCP26 and HF treatment.
Western blot data from collaborators has also demonstrated that NCP26 and Halofuginone elicit canonical ISR activation with GCN2 and eIF2$\alpha$ phosphorylation in a dose-dependent manner (figure \ref{fig:sup_western}).

Together this data shows that the ProRS inhibitors NCP26 and Halofuginone activate the amino acid starvation response in MM cell lines.
It also demonstrates that apoptotic pathways are activated following AAR activation, indicating that the cytotoxic effects of NCP26 and Halofuginone in MM cell lines are attributable in part to AAR activation and its downstream apoptotic mechanisms.
