\chapter{Bulk RNA-seq analysis of ProRS inhibitors}\label{ch:5-bulk}

%\minitoc

\section{Introduction}
Although MM treatment has improved significantly in the last 20 years, MM remains an incurable disease.
MM patients relapse and become resistant to drugs they have previously been treated with.
Therefore, research into novel therapeutics that can overcome multi-drug resistance and can be used to treat relapsed patients is of great importance.

New analogues of the drug Febrifugine are being actively researched for the treatment of many diseases, including numerous cancers\cite{halo2012clin, halo2012clin2}.
Febrifugine is the biologically active component of the herb \textit{Dichroa febrifuga}, which is considered one of the fundamental herbs in traditional Chinese medicine\cite{koepfli1949alkaloids}.
Febrifugine is a quinazolinone alkaloid, which has been shown to possess strong anti-malarial properties.
One such Febrifugine derivative, halofuginone (HF), has previously been shown to inhibit T Helper 17 (TH17) cell differentiation, by activating the amino acid response (AAR)\cite{sundrud2009halofuginone}.
Halofuginone inhibits the enzyme glutamyl-prolyl tRNA synthetase (EPRS)\cite{keller2012halofuginone}.
EPRS is a bifunctional aminoacyl-tRNA synthetase (AARS) and catalyses the the aminoacylation of glutamic acid and proline tRNA species (such that it charges its cognate tRNAs with glutamic acid and proline).
Halofuginone and Febrifugine compete with proline at the prolyl-tRNA synthetase active site of EPRS, specifically targeting utilisation of proline during translation\cite{keller2012halofuginone}.
This results in an accumulation of uncharged tRNA\textsuperscript{pro}s, giving the same cellular environment as if the cell were proline deficient, triggering the AAR to respond to the apparent proline deprived state.

AARSs are essential in protein synthesis, aiding in building chains of amino acids.
Human cancer cells often have an increased rate of protein synthesis, this is especially true in multiple myeloma, creating huge amounts of non-functional paraprotein, therefore are more reliant on aaRSs.
As discussed in Chapter \ref{ch:2-litreview}, HF has previously shown anti-MM activity in-vitro and in-vivo\cite{leiba2012halofuginone}.
HF induced cytotoxicity and apoptosis in numerous MM cell-lines and primary MM cells.
HF was also shown to inhibit MM growth and prolong survival in a mouse xenograft MM model.
However, the mechanism by which HF exerted its affect was not elucidated, and it is not clear if the AAR plays a role in HF's effectiveness in MM\@.

It has also been shown that HF's anti-MM effect can be reduced in the presence of excessive proline\cite{leiba2012halofuginone}.
Proline is very abundant in many tumours, which may result from upregulated matrix metalloproteinases (MMPs) degrading collagen in the extracellular matrix (ECM), and/or increased conversion of glutamine to proline compared to normal cells\cite{liu2013mirna}.
Elevated proline levels have been noted in the BM of MM patients\cite{fei2021metabolic}.
This increased proline concentration means that HF has a very narrow therapeutic window, and exhibits many side effects.

% Ralph drug drawing
\begin{figure}[ht]
    \centering
    \includegraphics[width=0.9\textwidth]{figures/Results/Transcriptomics/ralph_figure.pdf}
    \caption[Halofuginone and NCP26 structures]{Diagrams of halofuginone/MAZ1392 (a) and NCP26 (b) and their chemical structures.
    Halofuginone is an ATP dependent, proline and tRNA competitive ProRS inhibitor.
    NCP26 is an ATP competitive and proline uncompetitive ProRS inhibitor.
    Aminoacylation is an ATP-dependent process, requiring ATP to activate amino acids.
    Figure by Ralph Mazitschek.}
    \label{fig:ralph_diagrams}
\end{figure}

Recently, The Mazitschek group have synthesized numerous other compounds which target the ProRS site of EPRS.
One such example, NCP26 (Figure \ref{fig:ralph_diagrams}b), does not compete with proline for the active site of ProRS, unlike febrifugine and halofuginone.
NCP26 binds to the ATP binding site of ProRS, inhibiting utilization of ATP\@.
Aminoacylation is an ATP-dependent process, therefore blocking ATP binding inhibits this process, and also leads to an accumulation of uncharged tRNA\textsuperscript{pro}s.
NCP26 will hopefully alleviate some of the issues associated with HF treatment.
More ProRS inhibitors have been synthesized by the group, including NCP22.

This chapter uses MM cell-line models to assess the effectiveness of these ProRS inhibitors and uses bulk-RNA-seq to capture the transcriptional landscape following treatment, and determine their mechanism of action in MM\@.

% Flush figures
\clearpage

\section{Cell-based assay results}
\subsection{Cell line validation}
In this chapter, two MM cell lines are used.
Low-passage-number AMO-1 (henceforth referred to as wild type; WT) cells, which are sensitive to various MM treatments, and carfilzomib-resistant L363 (henceforth referred to as CFZr) cells.
Dose response curves for the two cell lines treated with proteasome inhibitors (PIs), carfilzomib and bortezomib, were generated to confirm the PI-resistance of CFZr cells and PI-sensitivity of WT cells (Figure \ref{fig:dose_carf_bort}).
% dose response figure
\begin{figure}[h]
\centering
\includegraphics[width=0.9\textwidth]{figures/Results/dose_response/carf_bort_wt_cfzr.pdf}
\caption[Carfilzomib and bortezomib dose response curves]{Multiple myeloma (MM) cell lines, AMO-1 (WT) and Carfilzomib-resistant L363 CFZr, treated with carfilzomib (CFZ) and bortezomib (BTZ).
MM cell lines were treated for 48 hours with a range of concentrations (approximately 1\si{\nano\Molar}-1\si{\micro\Molar}) of proteasome inhibitors (PI).
a) AMO-1 (WT cells).
b) Carfilzomib-resistant L363 cells (CFZr).}
\label{fig:dose_carf_bort}
\end{figure}
%%
The IC\textsubscript{50}s for AMO-1 (WT) cells were approximately 3.74\si{\nano\Molar} and 7.93\si{\nano\Molar} for bortezomib and carfilzomib, respectively.
For L363 CFZr cells, the IC\textsubscript{50}s were approximately 174.8\si{\nano\Molar} and 673\si{\nano\Molar} for bortezomib and carfilzomib, respectively.
This confirms that WT cells are very sensitive to proteasome inhibition, with IC\textsubscript{50}s in the low nanomolar range.
It also demonstrates that CFZr cells are much more resistant to proteasome inhibition, with the IC\textsubscript{50} of CFZr cells almost 85 times higher than WT cells for carfilzomib.

As expected, CFZr cells are more resistant to carfilzomib treatment than bortezomib treatment, as their resistance was developed by exposure to increasing carfilzomib concentrations.
However, it is clear that some cross-resistance to bortezomib is also acquired (the IC\textsubscript{50} for BTZ treatment is 46 times greater in CFZr cells than WT cells.)
This is likely due to the similar mechanism of actions of the two drugs, conferring resistance to bortezomib treatment too.
It could also be in part due to increased expression of general multi-drug resistant genes, such as ATP Binding Cassette Subfamily B Member 1 (\textit{ABCB1}).
This mimics clinical MM well, as patients are treated with carfilzomib as a second-line treatment, once they have already been treated with bortezomib, developed some resistance to it and relapsed.

Together, this validates that AMO-1 (WT) cells and L363 CFZr cells are sufficient cell models for PI-sensitive and PI-resistant MM\@.

\subsection{Halofuginone and NCP26 are cytotoxic to drug sensitive and drug resistant MM cell lines in a dose-dependent manner}
The effect of the ProRS inhibitors NCP22, NCP26, MAZ1805 (Halofuginol) and MAZ1392 (Halofuginone; HF) on cell viability was investigated using the MM cell lines AMO-1 and L363 CFZ-r.
The cell lines were treated with a range of concentrations of each compound (3.9\si{\nano\Molar}- 1\si{\micro\Molar}).
Cell viability was assessed using presto-blue assays (section \ref{subsec:method_doseresponse}), and dose response curves were generated (Figures \ref{fig:dose}a and \ref{fig:dose}c).
Halofuginone and NCP26 reduced cell viability of PI-sensitive AMO-1 cells and carfilzomib resistant L363 cells in a dose-dependent fashion.
For this concentration range, MAZ1805 and NCP22 seemed to have little effect on cell viability of WT or CFZ-r cells, and IC\textsubscript{50} values were unable to be calculated.
Halofuginone was found to be more potent than NCP26.
%
% dose response figure
\begin{figure}[h]
\centering
\includegraphics[width=\textwidth]{figures/Results/dose_response/MAZ_ncp_wt_cfz.pdf}
\caption[ProRS inhibitor dose response curves]{Multiple myeloma (MM) cell lines treated with ProRS inhibitors- MAZ1392 (Halofuginone), MAZ1805, NCP26 and NCP22.
MM cell lines were treated for 48 hours with a range of concentrations (3.91\si{\nano\Molar}-1\si{\micro\Molar}) of ProRS inhibitors.
a) and c) Dose response curves.
a and b) WT AMO-1 cells.
c) and d) Carfilzomib resistant L363 cells.
b) and d) Proportion of cells viable after 48 hours of 1\si{\micro\Molar} treatment with each agent.}
\label{fig:dose}
\end{figure}
%%

For WT AMO-1 cells, halofuginone had an IC\textsubscript{50} of 141.8nM and NCP26 an IC\textsubscript{50} of 502nM. For CFZ-r cells, halofuginone had an IC\textsubscript{50} of 1185nM and NCP26's IC\textsubscript{50} was ambiguous, a higher stock concentration would be required for calculation.
Figure \ref{fig:dose}b and \ref{fig:dose}d show the proportion of viable cells following 48 hours of treatment of the ProRS inhibitors.
WT AMO-1 cells were found to be more sensitive to NCP26 and halofuginone treatment than carfilzomib resistant L363 cells.
This may indicate some acquired cross-resistance built up from carfilzomib exposure or inherent resistance in the L363 cell line over the AMO-1 cell line.

\subsection{Carfilzomib and NCP26 have an additive or mild antagonistic effect together}\label{subsec:synergy}
Drug combinations have proved effective in MM in recent years, for example the combination of bortezomib, lenalidomide, and dexamethasone (VRd) is used extensively for newly diagnosed MM patients.
Drugs are often used in combination so that outcomes are improved (synergistic efficacy) or to reduce off-target effects and toxicity by minimizing the doses of the drugs (synergistic potency)\cite{meyer2019quantifying}.

To assess if NCP26 and carfilzomib work together synergistically, AMO-1 cells were treated with varying concentrations of NCP26 and Carfilzomib for 72 hours, then presto blue assays were performed to determine cell viability.
As shown by dose response curves in Figure \ref{fig:synergy}a and the response matrix in Figure \ref{fig:synergy}b, NCP26 and carfilzomib elicit a stronger cytoxic effect together than each agent individually.
SynergyFinder\cite{zheng2021synergyfinder} was used to calculate the compounds' synergy scores (-4.66 ZIP; -4.18 Loewe; -5.53 Bliss).
From these values it is unlikely that NCP26 and Carfilzomib work together synergistically.
NCP26 and carfilzomib seem to have an additive effect together, or slight antagonistic effect.
%%%
% synergy figure
\begin{figure}[h]
\centering
\includegraphics[width=0.9\textwidth]{figures/Results/Transcriptomics/synergy.pdf}
\caption[NCP26 and carfilzomib synergism]{Investigating potential synergy between NCP26 and carfilzomib.
AMO-1 cells were treated with varying concentration combinations of NCP26 and carfilzomib for 72 hours, then cell viability was determined using presto blue assays.
\textbf{A)} Dose response curves for NCP26 with different carfilzomib concentrations.
\textbf{B)} Matrix view of NCP26 and Carfilzomib concentration responses.
}
\label{fig:synergy}
\end{figure}
%%
%%%%%%% should this go in discussion
This reflects a previous result where HF demonstrated moderate antagonism with the bortezomib\cite{leiba2012halofuginone}.
This may indicate that HF and NCP26 may not work very well in combination with proteasome inhibitors in MM.
Although, the group only used a single concentration of BTZ (5\si{\nano\Molar}) and 3 concentrations of HF (25, 50 and 100\si{\nano\Molar}).
This limits their ability to determine synergy, as only very few concentration combinations were tested.
In the same study, HF was shown to exhibit moderate synergistic cytotoxicity with other anti-MM agents, including the IMiD lenalidomide and the corticosteroiod dexamethasone.
This might demonstrate the potential of using ProRS inhibitors in combination with IMiDs and corticosteroids, perhaps in place of proteasome inhibitors, once PIs become less effective for MM patients.
Indeed, with the promising results in Figure \ref{fig:dose}d, HF and NCP26 have shown cytotoxcity against PI-resistant cells and therefore hint towards being an effective agent against relapsed and PI-resistant MM\@.
%%%%%%%%%

% Flush figures
\clearpage

\section{ProRS inhibitor treatment bulk RNA-seq}

\subsection{Experiment overview}
The transcriptome of drug-sensitive and PI-resistant MM cells following treatment with four ProRS inhibitors was investigated using bulk RNA-seq.
PI-sensitive WT AMO-1 cells and carfilzomib resistant L363 cells (CFZr) were used.
Cells were treated for 6 and 24 hours with a DMSO control, or 1\si{\micro\Molar} of a ProRS inhibitor (MAZ1392 (Halofuginone), NCP26, NCP22 and MAZ1805 (Halofuginol)), or 100\si{\nano\Molar} carfilzomib (AMO-1 WT cells only).
CFZr cells were treated in the presence of 100\si{\nano\Molar} carfilzomib.
Samples were prepped for sequencing as in Section \ref{sec:bulk_lib_prep}.
The computational workflow for bulk RNA-seq analysis is outlined in Section \ref{subsec:bulk_data_pro}.

\subsection{Clustering}

% bulk clustering figure
\begin{figure}[p]
\centering
\includegraphics[width=0.9\textwidth]{figures/Results/Transcriptomics/bulk_clustering_maz_ncp_wt_cfz.pdf}
\caption[Bulk RNA-seq sample clustering]{Bulk RNA-seq sample clustering.
WT AMO-1 cells and CFZr L363 cells treated for 6 or 24 hours with a DMSO control, 100nM carfilzomib or 1uM of a ProRS inhibitor (MAZ1392/Halofuginone, MAZ1805/Halofuginol, NCP26 and NCP22).
Clustering analysis of sample-sample distances (a, c and e) and principal component analysis (PCA; b, d and f).
a) and b) both cell types (WT and CFZr) displayed; c) and d) WT AMO-1 only; e) and f) CFZr samples only.
}
\label{fig:clustering_bulk}\end{figure}
%%

Figure \ref{fig:clustering_bulk} shows clustering analysis.
Figure \ref{fig:clustering_bulk}a and \ref{fig:clustering_bulk}b show the samples distinctly separate into respective cell types, and this makes up the majority (69\%) of the variance in the dataset.
Therefore, the different cell types (WT and CFZr) were separated and analysed individually (Figures \ref{fig:clustering_bulk} c-f).

The less active compounds from the ProRS inhibitor dose response curves, MAZ1805 and NCP22, cluster closely with the DMSO-treated controls.
The more active inhibitors, NCP26 and halofuginone, cluster separately from DMSO controls and less active inhibitors, indicating they have elicited a stronger transcriptional response.
At 6 hours NCP26 and halofuginone cluster closely together.
At 24 hours NCP26 and halofuginone separate more.
This may suggest a distinction in their mechanism of action, or this could just be reflecting the differences in their potency.

In WT cells, carfilzomib clusters separately from DMSO and the less active inhibitors.
At 6 hours, carfilzomib samples are very separate from the NCP26 and halofuginone cluster, but they cluster together more closely at 24 hours.
This could suggest an initial difference in mechanism of action and transcriptional response to the ProRS inhibitors, but culminating in a similar response as time progresses, such as cell stress and cell death pathways.

% PCA pathway figure
\begin{figure}[htb]
\centering
\includegraphics[width=0.9\textwidth]{figures/Results/Transcriptomics/bulk_clustering_PCs_analysed.pdf}
\caption[PCA pathway enrichment]{PCA pathway enrichment.
Pathway enrichment analysis performed using REACTOME of top contributing genes from principal component analysis (PCA).
A) WT AMO-1 dataset with carfilzomib-treated samples removed.
PC1 seems to account for the separation between DMSO controls/ less active compounds and the more active ProRS inhibitors (NCP26 and halofuginone).
Genes contributing positively towards PC1 are upregulated in NCP26 and halofuginone compared to DMSO controls and less active compounds.
Enriched pathways from top genes in PC1 shown beneath PCA plot.
B) CFZr L363 dataset.
PC2 seems to account for the separation between DMSO controls/ less active compounds and the more active ProRS inhibitors (NCP26 and halofuginone).
Genes contributing negatively towards PC2 (down arrow), are upregulated in NCP26/ halofuginone compared to controls.
Pathways enriched from top genes in PC2 shown beneath PCA plot.}
\label{fig:pca_pathway}
\end{figure}
%%

The top principal components of the WT AMO-1 and CFZr datasets were examined more closely.
Carfilzomib-treated samples were removed from the WT dataset, to ensure that the difference between controls and active ProRS inhibitors was captured in PC1 or PC2 following dimensionality reduction
Differential expression, variance stabilising transformation and PCA was re-performed to the CFZ-less dataset.
Pathway enrichment analysis was performed for the top genes from the principal component accounting for the difference between controls and halofuginone/NCP26 treatment (PC1 for WT cells, PC2 for CFZr cells; Figure \ref{fig:pca_pathway}).
The pathways `ATF4 activates genes in response to ER stress', `Response of GCN2 to amino acid deficiency' and `cytosolic tRNA aminoacylation' were all enriched, suggesting that the amino acid response is activated.
The `unfolded protein response' (UPR) was also enriched, as well as `PERK regulates gene expression'.
The UPR is a member of the integrated stress response (ISR), so many genes involved in the UPR overlap with genes involved in the AAR\@.
\textit{PERK} regulates the translation response of the UPR\@.
Additionally, genes involved in the cell cycle and G1/S transition are negatively enriched.
Previously, HF has been shown to induce the accumulation of cells in the G\textsubscript{0}/G\textsubscript{1} cell cycle\cite{leiba2012halofuginone}.

\subsection{Drug sensitive MM}

\subsubsection{Differential expression}
For AMO-1 WT cells at 6 hours, 2119 genes were differentially expressed ($\lvert log_{2}FC \rvert$ > 0.5 and p\textsubscript{adj}<0.05; DE) for NCP26 treated-samples, 3019 DE genes (DEGs) for halofuginone, 33 DEGs for MAZ1805, 218 DEGs for NCP22, and 983 DEGs for carfilzomib-treated samples compared to DMSO controls.
At 24 hours, 3323 DEGs for NCP26-treated samples, 3426 DEGs for halofuginone, 2 DEGs for MAZ1805 and 2260 DEGs for carfilzomibtreated samples compared to DMSO controls.
DEGs for NCP26 and halofuginone treatment are shown in Figure \ref{fig:wt_de}.
%
% DE volcano scatter WT %
\begin{figure}[htb]
\centering
\includegraphics[width=0.9\textwidth]{figures/Results/Transcriptomics/WT/wt_de_volcano_scatter.pdf}
\caption[Differentially expressed genes WT AMO-1 cells]{Differentially expressed genes (DEGs) for halofuginone and NCP26 treated WT AMO-1 cells at 6 and 24 hours.
Scatter plot of genes for 6 hours treated vs 24 hours treated.
Red points indicate genes which are differentially expressed (p\textsubscript{adj} < 0.01) at both 6 and 24 hours.
Volcano plots are also shown.
Blue points indicate downregulated DEGs (p\textsubscript{adj} < 0.01 \& $\log_{2}FC < -1$).
Red points indicate upregulated DEGs (p\textsubscript{adj} < 0.01 \& $\log_{2}FC > 1$).
Top DEGs are annotated with HGNC symbols.
}
\label{fig:wt_de}
\end{figure}
%%
Genes highly differentially expressed by the ProRS inhibitors are coloured and annotated with their gene symbol.
Numerous genes involved in stress response pathways can be seen to be upregulated following NCP26 and halofuginone treatment, including \textit{TRIB3}, \textit{JUN} and \textit{ATF3}.
Additionally, various histone genes are upregulated.
Some genes involved in cell cycle progression, such as \textit{CDKN1A}, are downregulated following ProRS inhibition.

Transcriptional changes for 6 hours of exposure to NCP26 were compared with proteomic changes for the same treatment condition (Figure \ref{fig:proteomic_rna_scatter}).
Proteomic data was supplied by collaborators <COLLABORATORs names here>.
% scatter plot proteomics vs rna-seq %
\begin{figure}[htb]
\centering
\includegraphics[width=0.5\textwidth]{figures/Results/Transcriptomics/WT/rna_vs_proteomics_ncp26_6h_scatter.pdf}
\caption[]{Scatterplot of proteomic and RNA-seq datasets depicting changes after 6 hr NCP26 exposure to AMO-1 cells.
Red points indicate genes which are differentially expressed (p\textsubscript{adj} < 0.01) in both RNA-seq and proteomic datasets.
Folllowing NCP26 treatment, TRIB3 and INHBE (both \textit{ATF4} targets) were the proteins with the highest increase in abundance ($\log_{2}$FC = 0.64 and INHBE  $\log_{2}$FC = 0.58) in the proteomic dataset and both were significantly upregulated in the RNA-seq dataset.
I analysed proteomic data supplied by <COLLABORATORS>.
}
\label{fig:proteomic_rna_scatter}
\end{figure}
%%
%
Consistent with a mechanism of a global reduction of protein synthesis (with the exception of preferential translation of \textit{ATF4}) upon  ISR induction and eIF2$\alpha$ phosphorylation, very few proteins with an increased abundance were identified (52 proteins with a $\log_{2}$ fold-change between 0.2 and 0.64).
Additionally, a larger shift of proteins with lower abundance in NCP26 samples compared to DMSO was found.
This correlates with transcriptional data where more DEGs were downregulated than upregulated for all ProRS treatment conditions.
Also, in-fitting with this mechanism, selective \textit{ATF4} target genes are upregulated, such as \textit{TRIB3} and \textit{INHBE} (as seen highlighted in Figure \ref{fig:proteomic_rna_scatter}).
This shows a dominant role of the integrated stress response at 6 hours on the transcriptomic and proteomic level.

%\subsubsection{Pathway enrichment analysis}
%Pathway enrichment analysis was performed for the top DEGs for NCP26 and halofuginone treatment.
%REACTOME and Gene ontology biological processes (GOBP) pathways were explored.
%FIGURE OF PATHWAY analyses.

%%%%%%%%%%%%%%
%% AAR stuff
%%%%%%%%%%%%%%

The effects of NCP26 and halofuginone treatment on the amino acid starvation response were examined in more detail.
% AAR heatmap
\begin{figure}[p]
\centering
\includegraphics[width=0.5\textwidth]{figures/Results/Transcriptomics/WT/wt_halo_ncp26_aar_heatmap.pdf}
\caption[Amino acid starvation response genes heatmap WT cells]{Amino acid starvation response (AAR) genes heatmap for WT cells.
Differentially expressed genes (DEGs) from WT AMO-1 cells intersected with genes involved in the AAR.
A list of known AAR genes was compiled by collating AAR genesets from the Molecular Signatures Database (MSigDb).
The colour scale shows $\log_{2}$ fold change of expression for each treated sample, compared to its DMSO time control.
Red indicates an upregulated gene and blue a downregulated gene.
}
\label{fig:wt_aar_heatmap}
\end{figure}
%%
%
AAR genesets `GOBP response to amino acid starvation', `REACTOME response of EIF2AK4/GCN2 to amino acid deficiency' and `KRIGE amino acid deprivation' were collated from the Molecular Signatures Database (MSigDb), making up a list of 166 unique genes.

This list of AAR genes was intersected with DEGs for ProRS inhibitor-treatment (i.e NCP26 vs DMSO and halofuginone vs DMSO at 6 or 24 hours) and a heatmap was constructed (Figure \ref{fig:wt_aar_heatmap}).
The AAR transcription factors \textit{ATF3}, \textit{DDIT3} (CHOP), \textit{CEBPB} and \textit{CEBPG} are all markedly upregulated following NCP26 and halofuginone treatment.
Amino acid transporters, such as \textit{SLC7A11}, were also upregulated following ProRS treatment.
Genes coding for aminoacyl tRNA synthetases were also seen to be upregulated, such as \textit{WARS1}, \textit{SARS1} and \textit{CARS1}.
% ARS1 genes
\begin{figure}[htb]
\centering
\includegraphics[width=\textwidth]{figures/Results/Transcriptomics/WT/wt_aaRS1_barchart.pdf}
\caption[Cytoplasmic aaRS gene expression change- WT cells]{Cytoplasmic aaRS gene expression change.
Log2 fold change for ProRS inhibitor/carfilzomib treated samples compared to DMSO controls.
Stars indicate significance at adjusted p-value < 0.05.
\textit{EPRS1} (the target of NCP26 and halofuginone) is highlighted in red.
}
\label{fig:wt_ARS1}
\end{figure}
%%
In total, nine cytoplasmic aaRS genes were overexpressed (p\textsubscript{adj} <0.05) after 6 hours of ProRS inhibition, and 14 cytoplasmic aaRS genes were overexpressed after 24 hours (Figure \ref{fig:wt_ARS1}).
\textit{EPRS1} was upregulated following NCP26 and halofuginone treatment, at 6 and 24 hours.
\textit{QARS1}, \textit{KARS1} and \textit{FARSA} (coding for the alpha subunit of FARS1) were found to be the most downregulated cytoplasmic aaRS genes, following ProRS inhibition.
Mitochondrial aaRS gene expression changes following ProRS inhibition is shown in Figure \ref{fig:ARS2_barchart}.
% indicating tumour suppressive role??

A list of 287 unique genes activated by \textit{ATF4} was compiled from the genesets: `REACTOME ATF4 activates genes in response to endoplasmic reticulum stress' and `ATF4 Q2' (genes having at least one occurrence of the transcription factor binding site V.ATF4 Q2 in the regions spanning up to 4 kb around their transcription starting sites).
Figure \ref{fig:wt_atf4_heatmap} shows a heatmap for ATF4 activated genes and genes DE by NCP26 and halofuginone.
Numerous \textit{ATF4} targets can be seen to be differentially expressed following ProRS inhibition with NCP26 and halofuginone.
Of the 287 genes in the list, 165 genes were differentially expressed with NCP26 or halofuginone treatment.


% ATF4 heatmap
\begin{figure}[p]
\centering
\includegraphics[width=0.5\textwidth]{figures/Results/Transcriptomics/WT/wt_halo_ncp26_atf4_heatmap.pdf}
\caption[Heatmap of \textit{ATF4} activated genes for ProRS treated WT cells]{Heatmap of \textit{ATF4}activated genes for WT cells treated with NCP26 and halofuginone.
Differentially expressed genes (DEGs) from WT AMO-1 cells intersected with \textit{ATF4} activated genes.
A list of genes activated by the transcription factor \textit{ATF4} was compiled by collating genesets from the Molecular Signatures Database (MSigDb).
The colour scale shows $\log_{2}$ fold change of expression for each treated sample, compared to its DMSO time control.
Red indicates an upregulated gene and blue a downregulated gene.
}
\label{fig:wt_atf4_heatmap}
\end{figure}
%%

Taken together, it is clear that following HF and NCP26 treatment of AMO-1 MM cells, the amino acid starvation response is activated, mediated via the transcription factor \textit{ATF4} and culminates in ER stress and downstream apoptotic mechanisms.

%% vs carfilzomib
\subsubsection{ProRS inhibitors vs carfilzomib}
Since their first use in MM, over 20 years ago, the mechanism of action of proteome inhibitors has been extensively studied and well described by researchers\cite{nunes2017proteasome}.
Figure \ref{fig:clustering_bulk}d demonstrates similarities at 24 hours between the transcriptional effects of ProRS inhibitors and the proteasome inhibitor carfilzomib on AMO-1 cells.
It also demonstrates ProRS and carfilzomib-treated samples separation at 6 hours, highlighting differences in their initial mechanism of action.

The similarities and differences between ProRS inhibitors and carfilzomib were studied in more detail.
% carf vs proRS venn diagram, scatter and heat
\begin{figure}[p]
\centering
\includegraphics[width=\textwidth]{figures/Results/Transcriptomics/WT/carf_halo_ncp_comparison_figure.pdf}
\caption[ProRS inhibitors compared with Carfilzomib's mechanism of action]{ProRS inhibitors compared with Carfilzomib's mechanism of action.
a) Venn diagrams showing overlapping differentially expressed genes (DEGs; upregulated or downregulated following treatment)at 6 and 24 hours.
b) Scatter plots showing correlation of carfilzomib DEGs against halofuginone or NCP26 DEGs.
c) Pathway analysis (Gene ontology biological processes; GOBP) for top upregulated genes.
d) Heatmap of selected differentially expressed genes upon carfilzomib, halofuginone and NCP26 treatment.
}
\label{fig:wt_carf_compare}
\end{figure}
%%
As seen by x-y trends of scatter plots and number of overlapping genes in venn diagrams (Figure \ref{fig:wt_carf_compare}a and b), ProRS inhibitors and carfilzomib treatment are more similar at 24 hours, and share more DEGs (compared to DMSO controls), than at 6 hours.
Somewhat obviously, NCP26 and HF share more overlapping DEGs with eachother than with CFZ\@.
HF shares more overlapping DEGs with CFZ than NCP26.
This is likely to do with potency and dosing, where both HF and CFZ were used at approximately 10 times their IC\textsubscript{50} value.

Carfilzomib treatment of AMO-1 cells resulted in a pronounced induction of the heat shock response, changes to ubiquitin mediated processes and protein folding, in line with the well-defined cellular changes of proteasome inhibition.
At 6 hours, some ISR effectors, such as \textit{ATF3} and \textit{JUN}, are upregulated by both carfilzomib and NCP26/Halofuginone treatment.
%Suggesting that ProRS inhibitors may have a distinct enough mechanism of action to bypass proteasome resistance.

At 24 hours, CFZ and ProRS inhibitor treatment seem to culminate in similar end-stage stress, cell cycle changes and apoptotic mechanisms.
Both ProRS inhibition and CFZ treatment result in pathway enrichment of `cell cycle arrest', `negative regulation of cell proliferation', `positive regulation of apoptosis process' and `type 1 interferon signalling pathway'.
%Both HF and CFZ treatment is shown to result in enrichment of angiogenic pathways;
%this is consistent with other preclinical studies, whereby HF induced anti-angiogenic effects.
Figure \ref{fig:wt_carf_compare}d shows a heatmap of NCP26, halofuginone and CFZ for $\log_{2}FC$ compared to DMSO controls at 6 and 24 hours for selected genes, belonging to various pathways/ classes.
NCP26 and halofuginone demonstrate similar effects on myeloma markers as CFZ, such as downregulating the MM pathological marker \textit{SDC1/ CD138}.
CFZ is an established anti-MM therapy approved in the clinic, therefore this is promising for the effectiveness of ProRS inhibitors in MM.


% OVERLAPPING ER stress, SAME NICHE = why tyhey dont work synergistically. good replacement maybe
% Not different enough to work synergistically, but different enough to kill resistant MM

%%%%%%%%%%%%%%%%%%%%%%%%%%%%%%%%%%%%%%%%%%%%%%%%%%%%%%%%%%%%%%%%%%%%%%%%%%%%%%%%%%%%%%%%%%%%%%%%%%%%%%%%%%%%%%%%%%%%%%%%
%%%%%%% CFZ L363 %%%%%%%%%
%%%%%%%%%%%%%%%%%%%%%%%%%%%%%%%%%%%%%%%%%%%%%%%%%%%%%%%%%%%%%%%%%%%%%%%%%%%%%%%%%%%%%%%%%%%%%%%%%%%%%%%%%%%%%%%%%%%%%%%%
\afterpage{\clearpage}
\subsection{Carfilzomib-resistant cells}

\subsubsection{Differential expression}
For CFZr L363 cells at 6 hours, 1165 genes were DE for NCP26 treated-samples, 2424 DEGs for halofuginone, 222 DEGs for MAZ1805, and 0 DE genes for NCP22-treated samples compared to DMSO control samples.
At 24 hours, 852 DEGs for NCP26-treated samples, 256 DEGs for halofuginone, no genes were DE for MAZ1805 and NCP22 compared to DMSO controls.
%
% DE volcano scatter WT %
\begin{figure}[bht]
\centering
\includegraphics[width=0.9\textwidth]{figures/Results/Transcriptomics/CFzr/cfz_de_volcano_scatter.pdf}
\caption[Differentially expressed genes CFZr L363 cells]{Differentially expressed genes (DEGs) for halofuginone and NCP26 treated CFZr L363 cells at 6 and 24 hours.
Scatter plot of genes for 6 hours treated vs 24 hours treated.
Red points indicate genes which are differentially expressed (p\textsubscript{adj} < 0.01) at both 6 and 24 hours.
Volcano plots are also shown.
Blue points indicate downregulated DEGs (p\textsubscript{adj} < 0.01 \& $\log_{2}FC < -1$).
Red points indicate upregulated DEGs (p\textsubscript{adj} < 0.01 \& $\log_{2}FC > 1$).
Top DEGs are annotated with HGNC symbols.
}
\label{fig:cfz_de}
\end{figure}
%%
DEGs for NCP26 and halofuginone treatment are shown in Figure \ref{fig:cfz_de}.
Numerous genes involved in the AAR can be seen to be upregulated following NCP26 and halofuginone treatment, including \textit{TRIB3}, \textit{ATF3}, \textit{CHAC1}, \textit{INHBE}, \textit{PSAT1}, \textit{ASNS} and \textit{SESN2};
Amino acid transporters \textit{SLC7A11} and \textit{SLC6A9}, and tRNA aminoacyl synthetase \textit{WARS1} are also upregulated.

\subsubsection{Pathway enrichment analysis}
Pathway enrichment analysis was performed for the top DE genes for NCP26 and halofuginone treated samples compared to DMSO controls (Figure \ref{fig:cfz_pathway}).
Multiple pathways relating to endoplasmic reticulum stress and apoptosis were enriched following ProRS inhibition.
`Cellular repsonse to amino acid starvation' was enriched, along with `amino acid transport', indicating that NCP26/Halofuginone likely activate the amino acid starvation response in the CFZ resistant MM cells.
Additionally the pathway `response to unfolded protein' was enriched.
The unfolded protein response is part of the integrated stress response so shares many of the same effectors as the AAR, such as \textit{DDIT3/CHOP}.

% Pathway analysis %
\begin{figure}[htb]
\centering
\includegraphics[width=0.6\textwidth]{figures/Results/Transcriptomics/CFzr/CFZ_pathway_analysis-eps-converted-to.pdf}
\caption[Pathway analysis for ProRS inhibitor-treated CFZr cells]{Pathway analysis for ProRS treated CFZr cells at 24 hours.
Gene ontology biological processes (GOBP) was performed for the top DE genes.
}
\label{fig:cfz_pathway}
\end{figure}
%
The amino acid response was investigated more closely for the PI-resistant cell line.
Figure \ref{fig:cfz_aar_heatmap} shows a heatmap of DEGs for NCP26/Halofuginone treated CFZr cells at 6 or 24 hours, which are known members of the AAR (collated from MSigDbl- as above).

% AAR heatmap
\begin{figure}[p]
\centering
\includegraphics[width=0.5\textwidth]{figures/Results/Transcriptomics/CFzr/cfz_halo_ncp26_aar_heatmap.pdf}
\caption[Amino acid starvation response genes heatmap CFZr cells]{Amino acid starvation response (AAR) genes heatmap for CFZr cells.
Differentially expressed genes (DEGs) from CFZr AMO-1 cells intersected with genes involved in the AAR.
A list of known AAR genes was compiled by collating AAR genesets from the Molecular Signatures Database (MSigDb).
}
\label{fig:cfz_aar_heatmap}
\end{figure}
%%
A strong elicitation of the AAR can be seen following NCP26 and halofuginone treatment.
The transcription factors \textit{CEBPG}, \textit{CEBPB}, \textit{ATF3}, \textit{DDIT3} are strongly upregulated, along with amino acid transporter \textit{SLC7A11}.
Other AAR genes are strongly upregulated with ProRS inhibitor treatment, such as \textit{ASS1}, \textit{ASNS}, \textit{CHAC1}, \textit{PSAT1}, \textit{PPP1R15A/GADD34}and \textit{TRIB3}.

% ARS1 genes cfz
\begin{figure}[htb]
\centering
\includegraphics[width=\textwidth]{figures/Results/Transcriptomics/CFZr/cfz_aaRS1_barchart.pdf}
\caption[Cytoplasmic aaRS gene expression change- CFZr cells]{Cytoplasmic aaRS gene expression change CFZr cells.
Log2 fold change for ProRS inhibitor/carfilzomib treated samples compared to DMSO controls.
Stars indicate significance at adjusted p-value < 0.05.
\textit{EPRS1} (the target of NCP26 and halofuginone) is highlighted in red.
}
\label{fig:cfz_ARS1}
\end{figure}
%%
In total, 13 cytoplasmic aaRS genes were overexpressed (p\textsubscript{adj} <0.05) after 6 hours of ProRS inhibition, and 12 cytoplasmic aaRS genes were overexpressed after 24 hours in CFZr cells (Figure \ref{fig:cfz_ARS1}).
Like in WT cells, \textit{EPRS1} was upregulated following NCP26 and halofuginone treatment, at 6 and 24 hours.
\textit{WARS1} and \textit{GARS1} were strongly overexpressed following ProRS inhibition.
Similar to treatment of WT AMO-1 cells, \textit{QARS1}, \textit{KARS1}, \textit{FARSA} and \textit{VARS1} did not follow the trend and gene expression of these genes was found to be downregulated in ProRS inhibitor treated samples.
Perhaps this indicates a tumour suppressive role of these aminoacyl tRNA synthetases.

% ATF4 heatmap
\begin{figure}[p]
\centering
\includegraphics[width=0.5\textwidth]{figures/Results/Transcriptomics/CFzr/cfz_halo_ncp26_atf4_heatmap.pdf}
\caption[Heatmap of \textit{ATF4} activated genes for ProRS treated CFZr cells]{Heatmap of \textit{ATF4} activated genes for CFZr L363 cells treated with NCP26 and halofuginone.
\textit{ATF4} activated genes, which are differentially expressed following NCP26 or halofuginone treatment at 6 or 24 hours.
A list of genes activated by the transcription factor \textit{ATF4} was compiled by collating genesets from the Molecular Signatures Database (MSigDb).
The colour scale shows $\log_{2}$ fold change of expression for each treated sample, compared to the corresponding DMSO control.
Red indicates an upregulated gene and blue a downregulated gene.
}
\label{fig:cfz_atf4_heatmap}
\end{figure}
%%

A list of 287 unique genes activated by \textit{ATF4} was compiled from the genesets: `REACTOME ATF4 activates genes in response to endoplasmic reticulum stress' and `ATF4 Q2' (genes having at least one occurrence of the transcription factor binding site V.ATF4 Q2 in the regions spanning up to 4 kb around their transcription starting sites).
Figure \ref{fig:cfz_atf4_heatmap} shows a heatmap for ATF4 activated genes and genes DE by NCP26 and halofuginone in CFZr cells.
Numerous \textit{ATF4} targets can be seen to be differentially expressed following ProRS inhibition with NCP26 and halofuginone.
Of the 287 genes activated by \textit{ATF4} in the list, 157 genes were differentially expressed with NCP26 or halofuginone treatment.
This indicates that many \textit{ATF4} downstream targets are being mediated, indicating that \textit{ATF4} has high transcriptional activity.

CFZ-resistant L363 cells show a similar engagement of the AAR and activation of \textit{ATF4}-mediated genes as PI-sensitive AMO-1 cells, following NCP26/HF treatment.
Additionally many downstream apoptotic mechanisms are enriched, indicating the initiation of cell death by AAR activation.
Taken with the cytotoxicity dose response data, this demonstrates that ProRS inhibitors are capable of killing PI-resistant MM cell lines.
Thus targeting the ProRS could be a potential effective strategy in overcoming PI drug resistance in clinical MM\@.

\section{ProRS inhibitor resistance bulk RNA-seq}

\subsection{Experiment overview}
Bulk RNA-seq was performed to explore the transcriptional changes of MM cells which had become resistant to ProRS inhibition.
Dr James Dunford created a NCP26 resistant cell line (NCP26R) by continually exposing L363 cells to NCP26, and gradually increasing NCP26 concentration over time.
NCP26R cells at time of sequencing were growing in 1\si{\micro\Molar} NCP26, with an IC\textsubscript{50} of approximately 1.6\si{\micro\Molar}.
Normal L363 cells have an IC\textsubscript{50} of approximately 0.5\si{\micro\Molar} for NCP26.
Triplicates of normal L363 cells and NCP26R L363 cells were harvested and prepared for sequencing as in Section \ref{sec:bulk_lib_prep}.

\subsection{Results}
As expected, the samples clustered into two distinct groups: control L363 samples and NCP26R samples (Figure \ref{fig:sup_NCP26R_bulk}a and b).
The differences between the two cell groups made up most of the variance in the dataset (PC1 = 90\% overall variance).

4129 genes were differentially expressed ($\lvert log_{2}FC \rvert$ > 0.5 and p\textsubscript{adj}<0.05; DE) between NCP26R L363 samples and L363 controls (Figure \ref{fig:NCP26R_DE}a).
% DE volcano and aaRS
\begin{figure}[htb]
\centering
\includegraphics[width=\textwidth]{figures/Results/Transcriptomics/NCP26R/NCP26R_DE_volcano_bar.pdf}
\caption[NCP26R bulk RNA-seq differentially expression]{Differentially expressed genes (DEGs) for NCP26R L363 cells vs L363 controls.
a) Volcano plot. Blue points indicate downregulated DEGs (p\textsubscript{adj} < 0.01 \& $\log_{2}FC < -1$).
Red points indicate upregulated DEGs (p\textsubscript{adj} < 0.01 \& $\log_{2}FC > 1$).
Top DEGs are annotated with HGNC symbols.
b) Log2FC of cytoplasmic aaRS genes.
Stars indicate differential expression (p\textsubscript{adj}<0.05).
\textit{EPRS1} is upregulated in NCP26R cells.
}
\label{fig:NCP26R_DE}
\end{figure}
%%
The Log2FC for cytoplasmic aaRS genes is shown in Figure \ref{fig:NCP26R_DE}b.
Nine out of 20 aaRS genes were upregulated (p\textsubscript{adj} <0.05) in NCP26-resistant cells, including \textit{EPRS1}.
Four aaRS genes were downregulated (p\textsubscript{adj} <0.05), including \textit{QARS1} and \textit{VARS1}, which were also downregulated following ProRS inhibitor treatment.

Next, pathway enrichment analysis was performed.
% pathway GSEA figure

From pathway enrichment it is clear that NCP26R cells in overcoming NCP26 treatment, adaptions to ER pathways are involved.
% AAR heatmap NCP26R
\begin{figure}[p]
\centering
\includegraphics[width=0.7\textwidth]{figures/Results/Transcriptomics/NCP26R/NCP26R_aar_heatmap.pdf}
\caption[Amino acid starvation response genes heatmap- NCP26R cells]{Amino acid starvation response (AAR) genes heatmap for NCP26R cells.
Differentially expressed genes (DEGs) for NCP26R L363 cells vs control L363 cells intersected with genes involved in the AAR.
A list of known AAR genes was compiled by collating AAR genesets from the Molecular Signatures Database (MSigDb).
Normalised gene counts are shown for each sample.
}
\label{fig:ncp26R_aar_heatmap}
\end{figure}
%%
Transcriptional changes to the amino acid starvation response were explored.
47 AAR genes were differentially expressed (p\textsubscript{adj} <0.05) in NCP26R cells compared to controls (Figure \ref{fig:ncp26R_aar_heatmap}).
Many of the AAR genes seem to follow similar expression changes as with NCP26 treatment, such that many AAR genes that were upregulated (downregulated) following NCP26 treatment, were also upregulated (downregulated) in NCP26R cells.
For example, transcription factors \textit{ATF3}, \textit{DDIT3}, \textit{CEBPB} and \textit{CEPBG} are all overexpressed in NCP26R cells vs controls, and NCP26-treated cells vs DMSO-treated cells.
However, some AAR genes did not follow this pattern.
The master amino acid regulator, \textit{ATF4}, was downregulated in NCP26R cells, yet is upregulated following ProRS inhibitor treatment in CFZr cells.
Previously, it has been shown that \textit{ATF4} knockdown (but not \textit{ATF3}) has rendered tumour cells resistant to amino acid starvation\cite{cheng2018arginine}.
\textit{CHAC1} was also downregulated in NCP26R cells.
Overexpression of \textit{CHAC1} has been implicated in ER stress, oxidative stress and apoptosis\cite{crawford2015human}.
\textit{EIF2A} (coding for eiF2$\alpha$) is also downregulated in NCP26R cells.
Phosphorylation of eiF2$\alpha$ by GCN2 or PERK (sensors of AAR and UPR, respectively) results in ATF4 activation.
\textit{FAS}, which encodes TNF receptor superfamily member 6, a receptor involved in the induction of apoptosis, was downregulated in NCP26R cells.
Moreover \textit{CDKN1A} and \textit{MIOS} were downregulated in NCP26R cells compared to controls, yet are upregulated following ProRS inhibition in CFZr cells.
As mentioned, a well-known function of \textit{CDKN1A} is to arrest cell cycle progression, it is also though to be instrumental in the execution of apoptosis following caspase activation.
% Interestingly \textit{GCN1} expression was upregulated in NCP26R cells, yet was downregulated following ProRS inhibition in CFZr cells.

These transcriptional changes in NCP26R cells, indicate that greater ProRS inhibition is required in this cell line to overcome reduced EIF2A, CHAC1 and ATF4 expression and to tip the balance of ER stress from pro-survival to pro-death signalling.
These adaptations further corroborate the involvement of the AAR in the mechanism of action and effectiveness of ProRS inhibitors in multiple myeloma.
%NPRL2, tumour suppressor
%Nprl2, is a putative tumor suppressor gene that inhibits cell growth and enhances sensitivity to numerous anticancer drugs including cisplatin

GCN1 increased expression (maybe sense AA too??)- usually down.
GCN1 thought to be required for GCN2 activation and AAR induction\cite{kim2020aminoacyl}.

ASS1


% flush figures
\clearpage
\section{Summary}
NCP26 and halofuginone have been shown to reduce cell viability of drug sensitive and PI-resistant MM cell lines in a dose-dependent manner.
It has been shown that NCP26 and carfilzomib are not synergistically cytotoxic together.
This reflects a previous study where HF and bortezomib were found to be moderately antagonistic in combination.
Therefore, administering a PI and ProRS inhibitor in combination would likely offer no increased benefit to patients.
This is likely due to similar mechanisms of activating ER stress pathways.
However from the dose response data, it is clear that ProRS inhibitors are cytotoxic against PI-resistant MM cell lines.
Indicating that their mechanism of actions are still distinct enough that NCP26 and HF have an effect on CFZ resistant cells.
Together with the previous result of HF interacting synergistically with lenalidomide and dexamethasone\cite{leiba2012halofuginone}, this could support a case for using ProRS inhibitors as a replacement for PIs once patients have relapsed and stop responding to proteasome inhibition.

% From omics data, a general shift towards lower protein abundance and downregulation of gene expression has been demonstrated following exposure of MM cells to ProRS inhibitors.
%This indicates activation of components of the ISR to halt global protein synthesis (with the exception of \textit{ATF4} target genes).
%\textit{ATF4} target genes were shown to be upregulated and larger in abundance following NCP26 and halofuginone treatment.
%\textit{ATF4} is the master regulator of amino acid metabolism.
%It is activated by amino acid deprivation.
%Halofuginone and NCP26 treatment caused increased expression of \textit{ATF4} in both drug sensitive and PI-resistant MM cell lines.
%Expression of numerous genes involved in the AAR and ISR were markedly increased following halofuginone and NCP26 treatment.
%\textit{DDIT3} and other downstream pro-apoptotic genes were over expressed following NCP26 and HF treatment.
%Western blot data from collaborators has demonstrated that NCP26 and halofuginone elicit canonical ISR activation with GCN2 and eIF2$\alpha$ phosphorylation in a dose-dependent manner (Figure \ref{fig:sup_western}).
%Together this data shows that the ProRS inhibitors NCP26 and halofuginone activate the amino acid starvation response in MM cell lines.
%It also demonstrates that apoptotic pathways are activated following AAR activation, indicating that the cytotoxic effects of NCP26 and halofuginone in MM cell lines are attributable (in part) to AAR activation and its downstream apoptotic mechanisms.

\section{Discussion1}
This chapter has demonstrated using in-vitro MM cell models the capability of ProRS inhibitors to kill drug-sensitive and drug-resistant MM cells, and that the AAR is activated following ProRS inhibition.
ProRS inhibition activates \textit{ATF4} (the master regulator of amino acid metabolism) and upregulates its downstream targets, including aminoacyl tRNA synthetases, amino acid transporters, and pro-apoptotic mediators including \textit{DDIT3}.
A general shift towards lower protein abundance can also be seen following ProRS inhibitor treatment of MM cells, a hallmark of the amino acid starvation response to halt global protein synthesis.
It has also been demonstrated by collaborators that NCP26 and HF elicit GCN2 and eIF2$\alpha$ phosphorylation in a dose-dependent manner (Figure \ref{fig:sup_western}).
During amino acid depletion, GCN2 acts as a cellar sensor by binding uncharged tRNAs and autophosphorylating.
Activated GCN2 then phosphorylates eIF2-$\alpha$\cite{battu2017amino}.
Therefore, GCN2 and eIF2$\alpha$ phosphorylation indicates canonical activation of the amino acid starvation response by HF and NCP26.

\subsection{HF vs NCP26}
Halofuginone is an ATP-dependent, proline and tRNA competitive inhibitor of ProRS\@.
NCP26 is a proline non-competitive, ATP competitive inhibitor or ProRS\@.
Excess proline addition has been shown to have no effect on NCP26 treatment of MM cells (up to 20\si{\milli\Molar}).
Whereas HF's anti-MM effects have been shown to be abrogated by excess proline (Figure \ref{fig:sup_proline_excess}).
Cancer has been shown to adapt to therapy, changing its gene expression and genetic make-up to overcome drug treatment.
In MM, this can also be achieved by the ever-shifting balance of clonal populations, some of which possess advantageous mutations against specific therapies.
Therefore, metabolic reprogramming to increase intracellular proline levels would be a very simple way for cancer cells to adapt and overcome HF treatment.
As MM cells and the bone marrow already possess higher proline levels than normal cells, this could render the effectiveness of HF in MM very short-lived, and patients could become resistant very quickly.
NCP26 has demonstrated strong anti-MM effects, whilst not being susceptible to the metabolic liability of a proline supplementation resistance mechanism.
Additionally, NCP26 has demonstrated a stark difference between EC\textsubscript{50} values of AMO-1 cells and healthy PBMCs (greater than 10-fold).
The EC\textsubscript{50}s between HF and PBMCs were much closer in value.
This indicates that NCP26 will likely have a much wider therapeutic window than HF, and not exhibit as many of the dose-limiting toxicities and side-effects.
This would allow for higher dosing in patients and more effective killing of MM cells, and potentially kill even the most malignant and resistant subclones, limiting MRD and increasing time between remission and relapse.
NCP26 treatment should perhaps also be considered for diseases currently being treated with HF, such as Kaposi's sarcoma.

\subsection{PIs and amino acid starvation}
The capability of HF and NCP26 to overcome resistance in PI-resistant MM (rather than other MM therapies) was investigated due to the similarities in mechanisms of action of ProRS inhibition and proteasome inhibition.
Proteasome inhibition leads to fewer ubiquitin-tagged proteins being degraded, subsequently many misfolded proteins accumulate in the ER lumen.
This increases ER stress, activating the integrated stress response and causing the UPR to switch from a homeostatic, pro-survival system to a pro-apoptotic pathway\cite{kubiczkova2014proteasome, wallington2018resistance}.
The UPR shares many mediators with the the AAR, for example \textit{ATF4} and \textit{DDIT3}.
Therefore, it may be expected that the mechanisms overlap too much for ProRS inhibition to overcome PI-driven resistance mechanisms.
However, it was shown in-vitro that HF and NCP26 reduce the cell viability of carfilzomib-resistant L363 cells in a dose dependent manner (Figure \ref{fig:dose}).
The mechanisms must be sufficiently distinct for ProRS inhibition to kill PI-resistant cells.

Interestingly, amino acid depletion has been implicated in the mechanism of action of proteasome inhibitors\cite{suraweera2012failure}.
A study from 2012 demonstrated that treatment of mammalian cells with proteasome inhibitors MG-132 and bortezomib depleted the amino acid pool by decreasing the levels of asparagine/aspartate and cysteine to lethal levels.
Using western blots, PI treatment was shown to induce the ISR-- evident by eIF2$\alpha$ phosphorylation and expression of ATF4, DDIT3 and GADD34.
PI treatment was also shown to phosphorylate and activate GCN2.
Supplementation with cysteine and/or asparagine markedly increased cell viability following PI treatment.
Cysteine supplementation mostly abrogated the induction of the ISR, without affecting polyubiquitinated protein levels.
% -- indicating that the amino acid depletion is the major signal activating the ISR in PI treated cells.
Cysteine addition also largely reduced GCN2 phosphorylation, thereby establishing that a shortage of critical amino acids induces the ISR via GCN2 and the amino acid starvation response\cite{suraweera2012failure}.

Next, proteasome inhibition and amino acid depletion was investigated in \textit{Drosophila}\cite{suraweera2012failure}.
Bortezomib (50\si{\micro\Molar}) exposure alone to \textit{Drosophila} was lethal.
Bortezomib-treated flies supplemented with excess amino acids saw markedly increased survival, without affecting polyubiquitinated protein levels.
Taken together, it is clear that certain amino acids play a cytoprotective role upon proteasome inhibition, and that amino acid depletion is involved in the mechanism of action of PIs.
However, the group did not use any MM cell lines or primary patient samples in their study.
A higher concentration of PI is required to kill healthy cells or non-myeloma cells than myeloma cells, as there is lower baseline ER stress.
Therefore, the mechanism of action of PIs in myeloma cells may be different to that of healthy cells or different cancer cell lines.
For example, NF$\kappa$B signalling is well established in multiple myeloma pathology, and has been implicated in the mechanism of action of PIs; other cell lines without extensive NF$\kappa$B signalling may not react the same way to PI treatment as MM cells.
Additionally, due to the massive amount of paraprotein produced by MM cells, they already have increased strain on the proteasome and increased ER-stress.
It seems a large oversight to discuss a drug's mechanism of action, without using any cells derived from the diseases the drug is used to treat.
Furthermore, the group did not investigate other possible explanations for the induction of the ISR, for example they could have explored the expression of PERK, ATF6 or IRE-1, signal activators of the unfolded protein response, another branch of the ISR\@.
%The ISR could be activated by numerous converging signalling pathways, following PI treatment.

% It is an interesting finding that proteasome inhibition has been associated with amino acid depletion, however the study must be considered in context.
%A large oversight to discuss a drug's mechanism of action, without using cells originating from the diseases they are used to treat.
%Proteasome inhibitors have been approved for the treatment of multiple myeloma and mantle cell lymphoma, therefore it may be an overstated claim to discuss the mechanim of action of a drug without using cell lines for their major application in disease

It has previously been shown that during acute amino acid shortages, the proteasome degrades pre-existing proteins to supply amino acids for the synthesis of new proteins, and that proteasome inhibition under these conditions causes rapid AA depletion and markedly impaired translation\cite{vabulas2005protein}.
Mizrachy-Schwartz et al. (2010) showed that amino acid depletion of Cysteine and Methionine sensitized various cancer cells (including an MM cell line) to proteasome inhibition\cite{mizrachy2010amino}.
Amino acid depletion significantly decreased IC50s of bortezomib on the cancer cell lines, some of which bortezomib had very little effect on previously in complete growth medium.
% This further implicates the AAR in the mechanism of action of proteasome inhibitors.
The group hypothesized that amino acid depletion in combination with proteasome inhibition might improve the effectiveness of PIs in treating myeloma, and might even extend the drug class's applicability to a broader range of cancers.

In this chapter, the combination of PIs and ProRS inhibitors was investigated.
Despite evidence of PIs depleting the amino acid pool and amino acid depletion sensitizing cancer cell lines to proteasome inhibition\cite{suraweera2012failure, mizrachy2010amino}, carfilzomib and NCP26 were not found to act synergistically together.
They were found to have an additive effect, if not moderate antagonism together.
This supports a previous MM study where HF and bortezomib were found to be moderately antagonistic in combination\cite{leiba2012halofuginone}.
This may reflect that their mechanisms of action are too similar, whereby both inhibitor classes induce stress responses and pro-apoptotic mechanisms via DDIT3.
Perhaps this makes having both PIs and ProRS inhibitors in combination redundant.
Although, as we've seen, ProRS inhibitors are capable of overcoming PI-resistance in-vitro.
Therefore, the argument could be made for using ProRS inhibitors as a replacement for PIs in triplet therapies, once MM patients have amassed significant resistance to PIs.
Alternatively, as ProRS inhibitors are capable of overcoming PI-resistance, their additive effect together could be beneficial for maximising the number of MM subclones killed, limiting MRD, and prolonging time in remission.
% only mimics AA depletion.

\subsection{Targeting amino acid metabolism}
Proline metabolism has been thought of as great importance in tumourigenesis and MM pathology.
Although glutamine and glucose serve as the main metabolic substrates for tumour cells, proline as a stress substrate has gained lots of attention due to its unique metabolism, its availability in tumour microenvironments, and various stress-responses\cite{liu2013mirna}.
Proline, is the only proteogenic secondary amino acid and has its own metabolic system with its own family of enzymes, distinct from the usual amino acid-metabolising enzymes\cite{liu2013mirna}.
Proline metabolism is closely related with glutamine metabolism.
Cancer cells tend to convert more glutamine to proline compared to healthy cells.
Additionally, the bone marrow of MM patients is very proline-rich and hypoxic.
Hypoxia has been shown to increase proline production in-vitro.
Proline can be used by cancer cells as an energy source under conditions of nutrient stress\cite{phang2008metabolism}.

The conversion of glutamine to proline consists of several steps.
In the final step, pyrroline-5-carboxylate reductase (PYCR) catalyzes the conversion of pyrroline-5-carboxylate (P5C) into proline.
Three PYCRs have been identified: PYCR1, PYCR2 and PYCR3.
In a recent paper, Oudaert et al. (2022) investigated PYCR enzymes as potential therapeutic targets in MM\cite{oudaert2022pyrroline}.
The group found that PYCR1 and PYCR2 mRNA expression correlates with an inferior overall survival in MM, and MM cells from RRMM patients express significantly higher levels of PYCR1\@.
Small molecule PYCR1 inhibitor, pargyline, reduced proline production, MM cell viability and proliferation and increased apoptosis.
Pargyline increased bortezomib-mediated apoptosis in-vitro and enhanced bortezomib sensitivity to reduce tumor burden in a murine model.
However, pargyline as a single agent did not alter tumour load.
PYRC1 inhibition and silencing resulted in increased CHOP (protein encoded by \textit{DDIT3}) expression, and decreased protein synthesis-- suggesting the activation of integrated stress pathways, and possibly the AAR.
Inhibiting PYCR1 (one of the enzymes that converts glutamine into proline) reduces proline production and its concentration, this could result in proline depletion, and this amino acid depletion could be responsible for some of the cytotoxic effects from PYRC1 inhibition.
This could also explain why its effects are enhanced in combination with bortezomib, as amino acid depletion seems to sensitize cancer cells to proteasome inhibition\cite{mizrachy2010amino}.
This paper supports our work on ProRS inhibition (which mimics amino acid depletion), highlighting the effectiveness of targeting amino acid metabolism in MM, particularly proline metabolism.
However, the anti-MM cytotoxic effects seen from ProRS inhibition seem to be far greater than from PYCR1 inhibition as presented by Oudaert et al. (2022).
For example, NCP26 alone has shown significant anti-tumor activity in a MM mouse model, without significant body weight loss\cite{bottpreclinical2022}.
This could indicate the superiority of the activation of ISR pathways by mimicking proline depletion, whereas PYCR1 inhibition only reduces the concentration of proline sourced from P5C, and not dietary proline or the abundance of proline located in pre-existing proteins which can be catabolised by the proteasome.

In a manuscript currently under review, it was shown by collaborators that numerous inhibitors targeting ProRS, as well as other aaRSs induce significant anti-proliferative effects across multiple MM cell lines (Figure \ref{fig:sup_cell_line_aaRS})\cite{bottpreclinical2022}.
Borrelidin, a natural threonyl-tRNA synthetase (ThrRS) inhibitor and CysSA, an aminoacyl adenylate substrate analogue, both reduced proliferative activity of MM cell lines.
This shows that the anti-MM effects of Halofuginone and NCP26 may not be specific to only ProRS inhibition, and the anti-MM effects may extend to other aaRS inhibitors too.
Cheng et al. (2018) demonstrated that arginine starvation, induced typical ER stress and oxidative stress markers and killed tumour cells\cite{cheng2018arginine}.
These data further support aaRSs as targets of great importance in MM, and the rationale to develop inhibitors to target them.
A full screen of known aaRS inhibitors and substrate analogues against MM would be beneficial, to identify more agents capable of killing MM cells, some of which could be more effective against MM\@.
In \cite{suraweera2012failure}, MG-132 and bortezomib treatment caused depletion of asparagine/aspartate and cysteine levels, yet proline levels remained fairly constant.
It would be interesting to see if AspRS or CysRS inhibitors display any synergy with bortezomib.
However, this experiment was not conducted using MM cells.
Amino acid levels following PI-treatment of MM cell lines should be investigated first, to see which amino acids are depleted in MM following PI treatment.
This could present a whole host of new potential therapeutics in MM\@.

%PYCR2 expression upregulated following CFZ treatment.
%More detailed screens, see which AAs have an effect and work out why some. If as a effective as proline inhibition.
%Mitochondrial enzymes.. arginine starvation- mitochondrial genes and oxidateive stress in oxidative stress in our data

%protein catabolism contributes to the pool of free amino acids available for protein synthesis during cancer development, leading to a crucial role of the proteasome in cancer cell survival under conditions of nutrient depletion.\cite{mizrachy2010amino}

\section{Discussion}
Other therapeutics..

At time of writing, there were 292 active MM interventional clinical trials worldwide, with a further 441 studies recruiting or enrolling participants by invitation\cite{clinicaltrialsgov}.

In a co-authored manuscript currently under review, it has been shown that numerous inhibitors targeting ProRS, as well as other aaRSs induce significant anti-proliferative effects across multiple MM cell lines (Figure \ref{fig:sup_cell_line_aaRS}).
Borrelidin, a natural threonyl-tRNA synthetase (ThrRS) inhibitor and CysSA, an aminoacyl adenylate substrate analogue, both reduced proliferative activity of MM cell lines.
This shows that the anti-MM effects of Halofuginone and NCP26 may not be specific to only ProRS inhibitors, and this may extend to other aaRS inhibitors too.

TGF-beta increased in MM. HF affects TGF-beta signalling. bulk effects on TGF and smad?
% https://www.ncbi.nlm.nih.gov/pmc/articles/PMC1895802/#:~:text=TGF%2D%CE%B2%20signaling%20in%20multiple%20myeloma&text=TGF%2D%CE%B2%20is%20a%20potent,cell%20proliferation%20and%20immunoglobulin%20production.&text=In%20multiple%20myeloma%2C%20TGF%2D%CE%B2,from%20bone%20marrow%20stromal%20cells.

Significant anti-proliferative effects across all MM cell lines
inhibitors targeting ProRS
Other aaRS inhibitors or proline specifically? \ref{fig:sup_cell_line_aaRS}
Proline metabolism in cancer (https://www.intechopen.com/chapters/42308)
Proline metabolism as a target in MM
% (https://jeccr.biomedcentral.com/articles/10.1186/s13046-022-02250-3)
This study identifies PYCR1 as a novel target in bortezomib-based combination therapies for MM.



Proline levels unchanged or increased with PI treatment- but not MM cells.

detrimental to cells following proteasome inhibition

to a lethal level, and that this can be rescued following amino acid supplementation.
For example, Bortezomib treatment

proteasome inhibition results in a lethal amino acid shortage


Alternative to PIs? More effective and direct??

Target stress response via slightly different mechanisms once cancer adapts to overcome PI stress (UPR stress)
Cancer adapts

EPRS upregulated in patients, risk factor? role in disease?
GCN2 and GCN1\cite{kim2020aminoacyl}

Other results from paper, GCN2 sensed,

% Interferon signalling ?? myeloid stuff

in vivo NCP26 better than HF, not overly toxic. weight loss mice limited. HF narrow therapeutic window. Proline overcome
HFg covalent adducts???


We here show that both inhibitor chemotypes induce similar strong pro-apoptotic responses in myeloma cells by engagement of the ISR.
In contrast to the proline-competitive inhibitor HFG, NCP26 based inhibitors do not occupy the proline site in ProRS and rather are proline-uncompetitive, resulting in synergistic binding.
As our data show, this chemotype is not  affected by elevated cellular proline levels, which constitute a HFG resistance mechanism

Susceptibility to proline levels in the tumor microenvironment is an important concern in considering ProRS inhibition as therapeutic strategy in MM, since elevated proline levels have been noted in the MM bone marrow

Proline is the only proteinogenic secondary amino acid.
Proline requires its own family of enzymes, distinct from the usual amino acid-metabolising enzymes.
Proline metabolism is closely related with glutamine metabolism.
The conversion of glutamine to proline consists of several steps.
In the final step, pyrroline-5-carboxylate reductase (PYCR) catalyzes the conversion of pyrroline-5-carboxylate (P5C) into proline.
Three PYCRs have been identified: PYCR1, PYCR2 and PYCR3.


PYCR2 expression upregulated following CFZ treatment.
PYCR1 and PYCR2 mRNA expression correlate with an inferior overall survival in MM
RRMM patients express significantly higher levels of PYCR1.
PYCR1 inhibition reduced MM viability and proliferation and increased apoptosis
PYCR1 silencing reduced protein levels of p-PRAS40, p-mTOR, p-p70, p-S6, p-4EBP1 and p-eIF4E levels, suggesting a decrease in protein synthesis, which we also confirmed in vitro
pargyline, a PYCR1 small molecule inhibitor
Pargyline and siPYCR1 increased bortezomib-mediated apoptosis
combination therapy of pargyline with bortezomib reduced viability in CD138\textsuperscript{+} MM cells and reduced tumor burden in the murine mode \cite{oudaert2022pyrroline}.

PI resistance was investigated over IMiD due to similar mechanisms of action, ISR and GCN2 phosphorylation.

mTORC1. AA needed. SESN2 supresses