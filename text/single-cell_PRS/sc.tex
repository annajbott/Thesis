\chapter{\label{ch:6-sc}PRS inhibitors single-cell RNA-seq}

%\minitoc

\section{Introduction}
MM cells grow within the bone marrow and are supported as they grow by their microenvironment.
The MM microenvironment comprises a cellular compartment (composed of immune cells, endothelial cells, osteoblasts, osteoclasts and stromal cells) and a non-cellular compartment (composed of the extracellular matrix (ECM), cytokines, chemokines and growth factors)\cite{manier2012bone, kawano2015targeting}.
There are interactions between malignant plasma cells and the surrounding microenvironment.
The bone marrow microenvironment has been indicated to play a supportive role in migration, proliferation, differentiation and drug resistance of malignant plasma cells.
There is evidence linking the tumour microenvironment to progression of MGUS to active MM, for example significant matrix remodelling has been seen between the bone marrow of healthy individuals, MGUS and MM patients\cite{kawano2015targeting}.
Therefore, to get an accurate picture of MM, information must be acquired about the surrounding niche.

Historically, the tumour environment has been investigated following the isolation of populations of cells sorted from the tumour and then sequenced using traditional microarray or bulk RNA-seq techniques.
Bulk techniques measure the average expression across a sample, which is the sum of cell type specific expression weighted by cell type proportions.
Single-cell techniques measure expression for each individual cell and therefore provide information on clonal diversity that may be lost when pooling cells into bulk samples.
Furthermore, multiple myeloma is an extremely heterogeneous disease, this is seen both between patients and within an individual’s own tissue.
Applying single-cell techniques to capture the inter- and intra-individual heterogeneity is fundamental to identifying molecular and cellular signatures that define MM\@.


The advent of single-cell technologies has led to a better understanding of the complexity and diversity of the tumour microenvironment.
Seminal papers from Melnekoff et al. (2017)\cite{melnekoff2017single} and Ledergor et al. (2018)\cite{ledergor2018single} use scRNA-seq to reveal clonal transcriptomic heterogeneity in MM samples.
Melnekoff et al. (2017) demonstrated the clonal heterogeneity within MM using samples that were collected from eight relapsed MM patients.
The group performed t-SNE clustering analysis and the samples separated into eight transcriptionally distinct clones, each corresponding to a different patient.
This highlights the inter-patient differences of MM\@.
Ledergor et al. (2018) performed a similar study to evaluate clonal het- erogeneity within MM but also had a set of controls with which to compare the MM group.
They found that MM patients have greater inter-individual transcriptional variation, where each MM patient possessed a unique and individual plasma cell transcriptional program.
They also showed substantial intra-tumour heterogeneity (subclonal structures) of plasma cells in a third of their MM patient cohort.
These papers established the importance of using single-cell techniques to study MM, as to not miss the underlying clonal heterogeneity.
However both of these papers focussed soley on plasma cells and did not look at the surrounding bone marrow microenvironment.
To truly understand the complexities in MM and treatment of MM, interactions between plasma cells and the bone marrow niche must also be explored using single-cell techniques.


\section{Experiment overview}
Bone marrow samples from MM patients were treated with a DMSO control or six different compounds of interest.
The six compounds were: MAZ1392 (Halofuginone), NCP26, SGC-GAK, CL... (LOOK UP NAMES).
My analysis is only focussed on DMSO, Halofuginone and NCP26, however the other samples were included in the analysis and integrated clustering to increase  granularity.
Samples were treated with 1\si{\micro\Molar} of Halofuginone, NCP26 or DMSO control for 24 and 48 hours.

<2019 experiment overview>

\subsection{Analysis Overview}
The updated single cell analysis pipeline in section <ENTER section> was used to process the data.
In short, quality control and filtering of samples was performed, followed by clustering analysis and integration of all samples.
Next cell types were broadly annotated using the packages SingleR, Clustifyr and scClassify.
Myeloma cells were identified manually.

\section{Results}

\subsection{Quality control and filtering}
Halofuginone was used at 1\si{\micro\Molar}, a concentration almost 10 times its IC\textsubscript{50}.
Therefore at 48 hours many cells had been killed by Halofuginone treatment.

\subsection{Clustering}

\subsection{Integration}

\subsection{Annotation}

%% SCT UMAP annotation
\begin{figure}[htb]
\centering
\includegraphics[width=0.8\textwidth]{figures_made/SCT_umap_tsne_annotation.png}
\caption[Single-cell annotation]{SCT integration UMAP and tSNE annotated using SingleR, clustifyr and scClassify cell-type annotation packages.}
\label{fig:sc_annotation}
\end{figure}
%%

\subsubsection{Myeloma cell population identification}


\subsection{Composition}

