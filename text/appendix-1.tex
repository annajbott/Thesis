\chapter{Supplementary figures}

% western blot heatmap
\begin{figure}[ht]
\centering
\includegraphics[width=0.5\textwidth]{figures/appendix/western_gcn2_supplementary.pdf}
\caption[GCN2 and eIF2$\alpha$ western blot ]{Western blot demonstrating dose-dependent NCP26 canonical ISR activation with GCN2 and eIF2$\alpha$ phosphorylation. AMO-1 and MM.1S myeloma cell lines used.
Experiment performed by international collaborators on a multiple-lab collaborative research paper. <GROUP NAME HERE>
}
\label{fig:sup_western}
\end{figure}
%%

% Marker table
\begin{table}
    \centering
\begin{tabular}{|p{3cm}|p{9cm}|}
\hline
\textbf{Cell type}     & \textbf{Markers} \\ \hline
Multiple myeloma cells & CD138, CD38 (lower than plasma cells), SLAMF7,  BCMA, KRAS, IGKC, IGCL2. Reduced/ no CD20, CD19, CD45 expression.   \\ \hline
Normal plasma cells    & CD38, CD19, some BCMA \\ \hline
B cells                & CD20, some CD19 \\ \hline
T cells                &  TRAC, CD3D  \\ \hline
Cytotoxic cells        &  GZMH,  GZMB,  GZMA,  PRF1  \\ \hline
CD4+ T cells           & CCR7, SELL, TCF7 and T cell markers  \\ \hline
CD8+ T cells           & CD8A, cytotoxic markers and T cell markers  \\ \hline
NK cells               & KLRB1, KLRC1, KLRF1, CD16 and cytotoxic markers \\ \hline
Dendritic cells        & CD1C, FCER1A  \\ \hline
Monocytes              & CD14/CD16, CD68 \\ \hline
\end{tabular}
\caption[MM annotation gene marker expression]{Manual annotation markers for cell types originating from transcriptomic profiles of bone marrow samples.
SLAMF7, BCMA, KRAS, IGKC and IGCL2 are very highly expressed by MM cells, but are not exclusive to this cluster.
MM patient CD45\textsuperscript{+} immune cells scRNA-seq marker annotation can be found \cite{zavidij2019single}.}
\label{tab:annotation_markers}
\end{table}

% inferCNV
\begin{figure}[htb]
    \centering
    \includegraphics[width=\textwidth]{figures/Results/single_cell/data_processing/inferCNV_naive.pdf}
    \caption[inferCNV- newly-diagnosed MM]{InferCNV results for the newly-diagnosed MM dataset.
    [a and b] InferCNV heatmaps.
        The top panel shows expression values for the reference `normal' cells.
        The bottom panel shows expression values for the suspected malignant cells (clusters 2, 7 and 13), and other B-cell lineages (B cells and plasma cells).
        Red indicates chromosomal region amplifications and blue indicates chromosomal region deletions.
    a) De-noised inferCNV results.
    b) Hidden Markov-Model (HMM) copy number variation (CNV) region predictions.
        Only some chromosomal region gains predicted in MM clusters 2 and 13, and the plasma cell cluster.
    }
    \label{fig:inferCNV_naive}
\end{figure}
%