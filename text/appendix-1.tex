\chapter{Supplementary figures}

% ARS2 gene barcharts
\begin{figure}[h]
%1
\centering
\begin{subfigure}{\textwidth}
    %\centering
    \includegraphics[width=\textwidth]{figures/appendix/aaRS2_wt_barchart.pdf}
    \caption{WT AMO-1 cells}
\end{subfigure}
\medskip
%2
\begin{subfigure}{\textwidth}
    \includegraphics[width=\textwidth]{figures/appendix/aaRS2_cfz_barchart.pdf}
    \caption{CFZr cells}
\end{subfigure}
\caption[Mitochndrial aaRS gene expression change]{Mitochondrial aaRS gene expression change ProRS inhibitor/carfilzomib treated vs DMSO control.
Stars indicate significance at adjusted p-value < 0.05.}
\label{fig:ARS2_barchart}
\end{figure}

% western blot heatmap
\begin{figure}[h]
\centering
\includegraphics[width=0.5\textwidth]{figures/appendix/western_gcn2_supplementary.pdf}
\caption[GCN2 and eIF2$\alpha$ western blot ]{Western blot demonstrating dose-dependent NCP26 canonical ISR activation with GCN2 and eIF2$\alpha$ phosphorylation. AMO-1 and MM.1S myeloma cell lines used.
Experiment performed by international collaborators on a multiple-lab collaborative research paper\cite{bottpreclinical2022}.
}
\label{fig:sup_western}
\end{figure}
%%

% Proline supplementation
\begin{figure}[h]
\centering
\includegraphics[width=0.5\textwidth]{figures/appendix/excess_proline.pdf}
\caption[ProRS inhibitors excess proline]{ProRS inhibitors supplemented with excess proline.
AMO-1 cells treated with 500\si{\nano\Molar} of halofuginone or NCP26 for 48 hours in the presence of varying concentrations of proline (0, 1, 5, 10, and 20 \si{\milli\Molar}).
Experiment performed using triplicates.
Bars represent mean cell viability.
Experiment performed by international collaborators on a multiple-lab collaborative research paper\cite{bottpreclinical2022}.
}
\label{fig:sup_proline_excess}
\end{figure}
%%

% Heatmap MTT proliferation, MM cell lines other aaRS
\begin{figure}[h]
\centering
\includegraphics[width=0.5\textwidth]{figures/appendix/MM_cell_lines_other_aaRS.pdf}
\caption[aaRS inhibitors anti-proliferative activity in MM cell lines]{Anti-proliferative activities of aaRS inhibitors in MM cell lines.
Experiment performed by international collaborators on a multiple-lab collaborative research paper\cite{bottpreclinical2022}.
1\si{\micro\Molar} NCP26, NCP22, halofuginone, halofuginol (MAZ1805), borrelidin (threonyl-tRNA synthetase/ThrRS inhibitor), or 5 \si{\micro\Molar} L-ProSA, D-ProSA and CysSA (amino acid analogues).
72 hour MTT assay, n=2-5 independent experiments in triplicate technical repeats).
}
\label{fig:sup_cell_line_aaRS}
\end{figure}
%%

% Marker table
\afterpage{\clearpage}
\begin{table}[h]
    \centering
\begin{tabular}{|p{2cm}|p{6cm}|p{5cm}|}
\hline
\textbf{Marker type}     & \textbf{Expressed/ over-expressed}                                                                      & \textbf{Not expressed/ reduced expression} \\ \hline
Multiple myeloma cells & CD138, CD38 (lower than plasma cells), SLAMF7,  BCMA, KRAS, IGKC, IGCL2                                   & CD20, CD19, CD45                       \\ \hline
Normal plasma cells    & CD38, CD19, some BCMA                                                                                     & place                                    \\ \hline
B cells                & CD20, some CD19                                                                                           & b down                                   \\ \hline
T cells                &  TRAC, CD3D                                                                                               & cd4 down                                 \\ \hline
Cytotoxic cells        &  GZMH,  GZMB,  GZMA,  PRF1                                                                                & cd4 down                                 \\ \hline
CD4+ T cells           & CCR7, SELL, TCF7 and T cell markers                                                                       & cd4 down                                 \\ \hline
CD8+ T cells           & CD8A, cytotoxic markers and T cell markers                                                                & cd 8 down                                \\ \hline
NK cells               & KLRB1,  KLRF1 and cytotoxic markers                                                                       & nk down                                  \\ \hline
Dendritic cells        & dc up                                                                                                     & dc down                                  \\ \hline
Monocytes              & mono up                                                                                                   & mono down                                \\ \hline
etc...                 & else                                                                                                      & else down                                \\ \hline
\end{tabular}
\caption[Manual annotation markers]{Manual annotation markers for cell types originating from transcriptomic profiles of bone marrow samples.
SLAMF7, BCMA, KRAS, IGKC and IGCL2 are very highly expressed by MM cells, but are not exclusive to this cluster. }
\label{tab:annotation_markers}
\end{table}

% inferCNV
\begin{figure}[h]
    \centering
    \includegraphics[width=\textwidth]{figures/Results/single_cell/data_processing/inferCNV_naive.pdf}
    \caption[inferCNV- newly-diagnosed MM]{InferCNV results for the newly-diagnosed MM dataset.
    [a and b] InferCNV heatmaps.
        The top panel shows expression values for the reference `normal' cells.
        The bottom panel shows expression values for the suspected malignant cells (clusters 2, 7 and 13), and other B-cell lineages (B cells and plasma cells).
        Red indicates chromosomal region amplifications and blue indicates chromosomal region deletions.
    a) De-noised inferCNV results.
    b) Hidden Markov-Model (HMM) copy number variation (CNV) region predictions.
        Only some chromosomal region gains predicted in MM clusters 2 and 13, and the plasma cell cluster.
    }
    \label{fig:inferCNV_naive}
\end{figure}
%