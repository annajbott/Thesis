%% Table for treatment timeline
%
%
\begin{table}[hpt]
\centering
\begin{tabular}{|p{1cm}|p{2.8cm}|p{8.3cm}|p{1.28cm}|}
\hline
\textbf{Year} & \textbf{Treatment} & \textbf{Usage} & \textbf{Ref} \\ \hline
1958 & Melphalan & The alkylating agent melphalan was first used in plasma cell myeloma in 1958. & \cite{blokhin1958clinical} \\ \hline
1960s & Corticosteroids & Placebo-controlled double-blind trial of prednisone in multiple myeloma. Combinations of prednisone and melphalan showed an increased survival over melphalan alone. Dexamethasone and prednisone have become a cornerstone in the treatment of multiple myeloma. & \cite{mass1962comparison, alexanian1969treatment} \\ \hline
1980s & Stem-cell transplantations & Numerous successful allogenic and autologous bone marrow transplantations in patients with multiple myeloma &  \cite{mcelwain1983high, osserman1982identical, fefer1986identical, gahrton1987bone}  \\ \hline
2003 & Proteasome inhibitors & Bortezomib (BTZ), a first-in-class proteasome inhibitor (PI), was first approved by the FDA for use in RRMM. In 2008 it was approved for patients with no prior treatment. Carfilzomib (CFZ) was approved in 2012 for advanced MM and later in 2015 for treatment of relapsed MM. The oral PI, ixazomib, was approved as a combination treatment with lenalidomide and dexamethasone in 2016 for people who have received at least one previous treatment. & \cite{kane2003velcade,richardson2003phase,katsnelson2012next} \\ \hline
2006 & IMiDs & The anti-tumour activity of thalidomide was demonstrated in 1999, this led to the development of lenalidomide, the first approved immunomodulatory imide drug (IMiD) for use in multiple myeloma. Currently, thalidomide, lenalidomide and pomalidomide are approved for use in multiple myeloma & \cite{singhal1999antitumor,label47revlimid,san2013pomalidomide} \\ \hline
2015 & Monoclonal antibodies & In 2015, daratumumab (anti-CD38) and elotuzumab (anti-SLAMF7) monoclonal antibodies, were approved for MM treatment. & \cite{lokhorst2015targeting,lonial2015elotuzumab} \\ \hline
2020 & Antibody-drug conjugates & In 2020, Belantamab mafodotin-blmf (Blenrep), was approved by the FDA for use in MM patients who have already received 4 other treatments. Blenrep is an anti-BCMA monoclonal antibody linked to a chemotherapy drug. & \cite{fdaselinexor, podar2020selinexor} \\ \hline
2020 & Nuclear export inhibitors & In 2020, Selinexor was approved by the FDA for use in combination with BTZ and dexamethasone for MM patients who have had at least one prior therapy.& \cite{fdaselinexor, podar2020selinexor} \\ \hline
\end{tabular}
\caption[Timeline of treatment options for multiple myeloma]{Timeline of treatment options for multiple myeloma. Listed by first usage or FDA approval for MM.}
\label{tab:treatment_history}
\end{table}
%
% https://www.ncbi.nlm.nih.gov/pmc/articles/PMC5282737/
% https://www.ncbi.nlm.nih.gov/pmc/articles/PMC2265446/
% Panobinostat	HDACi
% Liposomal doxorubicin	DNA inter-calator