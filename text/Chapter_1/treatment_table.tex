%% Table for treatment timeline
%
%
\begin{table}[hpt]
\centering
\begin{tabular}{|p{1cm}|p{2.6cm}|p{8.5cm}|p{1.28cm}|}
\hline
\textbf{Year} & \textbf{Treatment} & \textbf{Usage} & \textbf{Ref} \\ \hline
1958 & Melphalan & An alkylating agent, first used in MM in 1958. & \cite{blokhin1958clinical} \\ \hline
1960s & Corticosteroids & Prednisone trialled in MM. Combination of prednisone and melphalan showed an increased survival over melphalan alone. Dexamethasone and prednisone have become standard in the treatment of MM. & \cite{mass1962comparison, alexanian1969treatment} \\ \hline
1980s & Stem-cell transplantations & Successful transplantations in patients with MM. MM patient gets high-dose chemotherapy to kill cells in the BM and then receives new, healthy stem cells. Autologous (patient's own stem cells) transplants are standard treatment for any eligible patients. Allogenic (another person's stem cells) transplants are riskier, more toxic and not standard in clinical practice. &  \cite{mcelwain1983high,osserman1982identical,fefer1986identical,gahrton1987bone,stemcellMM2022}  \\ \hline
2003 & Proteasome inhibitors & Bortezomib (BTZ), a first-in-class PI, was first approved by the FDA for use in RRMM. 2008- approved for naive patients. Carfilzomib (CFZ) was approved in 2012 for advanced MM and in 2015 for treatment of relapsed MM. The oral PI, ixazomib, was approved as a combination treatment with lenalidomide and dexamethasone in 2016 for people who have received at least one previous treatment. & \cite{kane2003velcade,richardson2003phase,katsnelson2012next} \\ \hline
2006 & IMiDs & The anti-tumour activity of thalidomide was demonstrated in 1999, this led to the development of lenalidomide, the first approved immunomodulatory imide drug (IMiD) for use in MM. Currently, thalidomide, lenalidomide and pomalidomide are approved for use in MM & \cite{singhal1999antitumor,label47revlimid,san2013pomalidomide} \\ \hline
2015 & Monoclonal antibodies & Daratumumab, isatuximab (both anti-CD38) and elotuzumab (anti-SLAMF7; not approved in the UK) monoclonal antibodies have been approved for use in MM. & \cite{lokhorst2015targeting,lonial2015elotuzumab} \\ \hline
2020 & Antibody-drug conjugates & In 2020, Belantamab mafodotin-blmf (Blenrep), was approved by the FDA for use in MM patients who have already received 4 other treatments. Blenrep is an anti-BCMA monoclonal antibody linked to a chemotherapy drug. & \cite{fdaBlenrep, markham2020belantamab} \\ \hline
2020 & Nuclear export inhibitors & In 2020, Selinexor was approved by the FDA for use in combination with BTZ and dexamethasone for MM patients who have had at least one prior therapy (not approved in the UK).& \cite{fdaselinexor, podar2020selinexor} \\ \hline
\end{tabular}
\caption[Timeline of treatment options for multiple myeloma]{Timeline of treatment options for multiple myeloma (MM). Listed by first usage or approval for MM in any country. USA FDA approved drugs not all approved yet in the UK. PI= proteasome inhibitor, BM= bone marrow, RRMM= relapsed and refractory MM.}
\label{tab:treatment_history}
\end{table}
%
% https://www.ncbi.nlm.nih.gov/pmc/articles/PMC5282737/
% https://www.ncbi.nlm.nih.gov/pmc/articles/PMC2265446/
% Panobinostat	HDACi
% Liposomal doxorubicin	DNA inter-calator

% In a stem cell transplant, the patient gets high-dose chemotherapy to kill the cells in the bone marrow. Then the patient receives new, healthy blood-forming stem cells.
%Stem cell transplant is commonly used to treat multiple myeloma. Before the transplant, drug treatment is used to reduce the number of myeloma cells in the patient’s body.
%
%For an autologous stem cell transplant, the patient’s own stem cells are removed from his or her bone marrow or peripheral blood before the transplant. The cells are stored until they are needed for the transplant. Then, the person with myeloma gets treatment such as high-dose chemotherapy, sometimes with radiation, to kill the cancer cells. When this is complete, the stored stem cells are given back to the patient into their blood through a vein.
%
%This type of transplant is a standard treatment for patients with multiple myeloma.
%
%In an allogeneic stem cell transplant, the patient gets blood-forming stem cells from another person – the donor.
%Allogeneic transplants are much riskier than autologous transplants
%At this time, allogeneic transplants are not considered a standard treatment for myeloma