%Standardised robust wet-lab and computational single-cell workflows will be established to characterise drug-resistant MM cells

Multiple myeloma (MM) is an incurable cancer of plasma cells, with an average five-year survival rate of approximately 50\%.
Over the last two decades application of now standard MM therapeutics, namely proteasome inhibitors (PI) and immunomodulatory imide drugs (IMiD), have almost doubled median survival time of MM patients.
However, most patients relapse and become resistant to drugs they previously have been treated with.
Acquired anti-cancer drug resistance remains one of the biggest barriers in the treatment of myeloma.
Therefore, identifying novel therapeutics effective against MM, which are capable of overcoming drug resistance is of the utmost importance.
Recently the prolyl-tRNA synthetase inhibitor, Halofuginone, has been shown to be effective against cancer, including one study demonstrating effectiveness against MM\@.





% old shit rip
%Recently, epigenetic mechanisms have been implicated in both the onset of MM and in the development of drug resistance.
%This thesis aims to investigate the changes that drive proteasome inhibitor drug resistance and to identify epigenetic compounds capable of reversing the resistance phenotype, and characterise their mechanism of action.
%The model MM cell line, AMO-1 was used in this work.
%Following an epigenetic compound library screen and bulk RNA-seq, a dual TRIM24/BRPF inhibitor (TRIM24i) was selected to be investigated further as it was shown to kill carfilzomib-resistant AMO-1 cells (aCFZ) in the presence of carfilzomib but had little effect on PI-sensitive (WT) AMO-1 cells, demonstrating that it is capable of re-sensitizing carfilzomib-resistant AMO-1 cells to carfilzomib.
%Transcriptomic, epigenomic and proteomic changes were studied using an array of `omic' techniques, including bulk and single-cell RNA-Seq, phosphoproteomics, ubiquitinomics, total proteomics, CyTOF and ChIP-Seq (PROBABLY will at some point).

