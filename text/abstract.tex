%Standardised robust wet-lab and computational single-cell workflows will be established to characterise drug-resistant MM cells

Multiple myeloma (MM) is an incurable cancer of plasma cells, with an average five-year survival rate of approximately 50\%.
More novel therapeutics, namely proteasome inhibitors (PI) and immunomodulatory imide drugs, have almost doubled median survival time of MM patients.
However, most patients relapse and become resistant to drugs they previously have been treated with.
Acquired anti-cancer drug resistance remains one of the biggest barriers in the treatment of myeloma.
Recently, epigenetic mechanisms have been implicated in both the onset of multiple myeloma and in the development of drug resistance.
This thesis aims to investigate the changes that underlie proteasome inhibitor drug resistance and to identify epigenetic compounds capable of reversing the resistance phenotype, and characterise their mechanism of action.

%The model MM cell line, AMO-1 was used in this work.
Following an epigenetic compound library screen, a TRIM24 inhibitor (TRIM24i) was selected to be investigated further as it was shown to kill carfilzomib-resistant AMO-1 cells (aCFZ) in the presence of carfilzomib but had little effect on PI-sensitive (WT) AMO-1 cells, demonstrating that it re-sensitizes AMO-1 cells to carfilzomib.
Bulk RNA-Seq and single cell RNA-Seq were employed to identify transcriptomic changes between WT and aCFZ cells, and between aCFZ cells and aCFZ cells treated with TRIM24. A number of proteomic methods were also carried out, including CyTOF, and phosphoproteomics and ubiquitinomics LC-MS/MS.
