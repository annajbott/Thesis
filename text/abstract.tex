%Standardised robust wet-lab and computational single-cell workflows will be established to characterise drug-resistant MM cells

Multiple myeloma (MM) is an incurable cancer of plasma cells, with an average five-year survival rate of approximately 50\%.
More novel therapeutics, namely proteasome inhibitors (PI) and immunomodulatory imide drugs, have almost doubled median survival time of MM patients.
However, most patients relapse and become resistant to drugs they previously have been treated with.
Acquired anti-cancer drug resistance remains one of the biggest barriers in the treatment of myeloma.
Recently, epigenetic mechanisms have been implicated in both the onset of multiple myeloma and in the development of drug resistance.
This thesis aims to investigate the changes that underlie proteasome inhibitor drug resistance and to identify epigenetic compounds capable of reversing the resistance phenotype, and characterise their mechanism of action.

%The model MM cell line, AMO-1 was used in this work.
An epigenetic compound library screen (consisting of 144 compounds) of PI-sensitive (WT) AMO-1 cells, carfilzomib-resistant AMO-1 cells (aCFZ) and bortezomib-resistant AMO-1 cells (aBTZ) revealed six compounds which killed PI-resistant AMO-1 cells in the presence of PI, but had no effect on WT cells.
Bulk RNA-Seq was carried out to explore the transcriptomic changes between WT and aCFZ cells, and also to explore the changes between aCFZ cells and aCFZ cells treated with the six epigenetic inhibitors. The compo

identify transcriptomic changes between PI-sensitive and PI-resistant AMO-1 cells and explore treatment with possible candidates for reversing resistance. TRIM24i

