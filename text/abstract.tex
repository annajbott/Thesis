\textbf{Background:} Multiple myeloma (MM) is an incurable cancer of plasma cells. Novel therapeutics, including proteasome inhibitors (PI) and immunomodulatory imide drugs, have almost doubled median survival time of MM patients. However, most patients relapse and become resistant to drugs they previously have been treated with. Acquired anti-cancer drug resistance remains one of the biggest barriers in the treatment of myeloma. 
    
\noindent
\textbf{Aims:} PI resistance mechanisms in MM will be investigated with the aim of reversing the resistance phenotype, making MM cells sensitive to proteasome inhibition. Standardised robust wet-lab and computational single-cell workflows will be established to characterise drug-resistant MM cells and their surrounding microenvironment at different points in disease progression. 
    
\noindent
\textbf{Results:} Cured cancer mate