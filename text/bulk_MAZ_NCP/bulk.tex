\chapter{\label{ch:5-bulk}PRS inhibitors bulk RNA-seq}

%\minitoc

\section{Introduction}
Although MM treatment has improved significantly in the last 10-20 years, still MM is an incurable disease.
Most MM patients relapse and become resistant to drugs they have previously been treated with.
Therefore, research into novel therapeutics that can overcome multi-drug resistance and can be used to treat relapsed patients is of great importance.
A new exciting class of compounds in treating MM have been derived from Febrifugine.
Febrifugine was first isolated from the Chinese herb Dichroa febrifuga, considered and important herb in traditional Chinese medicine, shown to have antimalarial effects.
One such derivative, Halofuginone, has been shown to inhibit T Helper 17 (TH17) cell differentiation, by activating the amino acid response (AAR)\cite{sundrud2009halofuginone}.
Halofuginone inhibits the enzyme glutamyl-prolyl tRNA synthetase (EPRS).
EPRS is a bifunctional aminoacyl-tRNA synthetase and catalyses the the aminoacylation of glutamic acid and proline tRNA species (i.e. it charges tRNAs with glutamic acid and proline).
Halofuginone and Febrifugine compete with proline at the prolyl-tRNA synthetase active site of EPRS, specifically targeting utlisiation of proline during translation\cite{keller2012halofuginone}.
This results in an accumulation of uncharged prolyl-tRNAs, giving the same cellular environment as if the cell were proline deficient, triggering the AAR to respond to the apparent proline deprived state.

Multiple Febrifugine derivatives have been synthesized.
% Dose response curves???

\subsection{Experiment overview}
The treatment of drug sensitive and carfilzomib resistant MM cell-lines with four compounds that inhibit  the  prolyl-tRNA  synthetase  active  site  of  EPRS  was investigated using bulk RNA-seq.
PI-sensitive WT AMO-1 cells and carfilzomib resistant L363 cells (check batch 1 type- are they L363 or AMO-1) were treated with 1\si{\micro\Molar} of MAZ1392 (Halofuginone), NCP26, NCP22, MAZ1805 or a DMSO control for 6 and 24 hours.


\section{Bulk RNA-seq of drug-sensitive AMO-1 cells treated with PRS inhibitors}




\section{Bulk RNA-seq of carfilzomib resistant cells treated with PRS inhibitors}



\section{Difference between Halofuginone and NCP26}

