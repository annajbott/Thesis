\chapter{\label{ch:2-litreview}Literature review: aminoacyl-tRNA synthetases and halofuginone}

\section{Introduction}
Aminoacyl-tRNA synthetases (aaRS) are a highly-conserved family of enzymes, responsible for ``charging'' tRNAs with their cognate amino acid\cite{pouplana2020}.
Human cytoplasmic aaRSs are either ``free'' as individual species or bound in a macromolecular complex, comprised of eight aaRSs and three auxiliary proteins, known as the multi-tRNA synthetase complex (MSC)\cite{pouplana2020}.
On top of their canonical enzymatic role, aaRSs also engage in non-enzymatic functions in numerous pathways, including angiogenesis, inflammation and metabolism.
Often species are released from the MSC to regulate these non-canonical activities\cite{kim2019evolution}.
aaRSs are involved in numerous diseases, including cancer.
Initially, due to their high fidelity and complex evolution over millennia, aaRSs were seen as an attractive drug target for antimicrobials, to enable specifically targeting microbial aaRSs with minimal effects on human cells.
The mechanism of action of febrifugine (FF), a quinazoline alkaloid that has long been used as an antimalarial remedy in Chinese medicine, has recently been revealed; it acts as a competitive inhibitor of ProRS (part of the bifunctional glutamyl-prolyl-tRNA synthetase enzyme; EPRS), responsible for charging tRNA\textsuperscript{pro} with proline.
Although it has potent antimalarial effects, FF exhibits liver and gastrointestinal (GI) toxicity, so cannot be used as a widespread drug, therefore several analogues of FF were developed in the hope of minimizing toxicity to the host`s cells.
One such analogue, halofuginone (HF) was synthesized and was shown to have the most potent antimalarial properties of all the derivatives, with lower toxicity to the host than FF, but still some liver toxicity and GI side effects remain\cite{zhang2017novel}.
Halofuginone has been applied to and showed promise in many other non-parasitic diseases too.
It has received orphan drug status for scleroderma and HIV-Related Kaposi's Sarcoma\cite{pines2015halofuginone}.
Recently, halofuginone`s application in various cancers has become of great interest, including but not limited to: metastatic brain tumours, bladder carcinomas, prostate cancer, renal carcinomas, hepatocellular carcinomas, lung cancer, breast cancer and multiple myeloma\cite{abramovitch2004halofuginone, elkin1999inhibition, gavish2002growth, nagler2004suppression, demiroglu2020anticarcinogenic, leiba2012halofuginone}.

This review will introduce the structure and function of (eukaryotic) aminoacyl-tRNA synthetases, provide an insight into their role in pathology and potential as therapeutic targets.
aaRSs inhibitors and their application in disease will be explored, focusing on the usage of the prolyl-tRNA synthetase inhibitor, halofuginone, in multiple myeloma.

\section{Function and structure of aminoacyl-tRNA synthetases}

aaRSs are an ancient family of ubiquitous enzymes, conserved across three major domains of life.
They can be traced back prior to the ``Last Universal Common Ancestor'' (LUCA)\cite{de2020evolution}.
aaRSs are essential for protein biosynthesis, and catalyse the first step in translation (Section \ref{subsec:translation}).
aaRSs catalyse the charging of tRNAs with their cognate amino acid.
This is a two-step process.
Firstly, aaRSs catalyse the formation of an aminoacyl-adenylate (activated amino acid) from their corresponding amino acid and an ATP molecule, releasing an inorganic pyrophosphate.
Next, aaRSs catalyse the reaction between the aminoacyl-adenylate and their cognate tRNA to release an AMP molecule and generate an aminoacyl (charged)-tRNA, ready to be used by the ribosome to decode mRNA (Equation \ref{eqn:aminoacyl}).
An example of this process would be prolyl-tRNA synthetase (abbreviated to ProRS) charging tRNA\textsuperscript{pro} with proline.

\begin{equation}\label{eqn:aminoacyl}
\begin{gathered}
aa + ATP \rightleftharpoons  aa.AMP + PPi \\
aa.AMP + tRNA^{aa} \longrightarrow  aa.tRNA^{aa} + AMP
\end{gathered}
\end{equation}

Eukaryotes have 20 cytoplasmic aaRS and 20 nuclear-encoded mitochondrial aaRS.
These are localised in distinct cellular compartments.
aaRSs are often denoted by their one letter amino acid symbol, followed by ARS and either 1 (indicating they are cytoplasmic) or 2 (indicating they are mitochondrial), for example LARS1 for cytoplasmic Leucyl-tRNA synthetase (LeuRS).
This review will focus on cytoplasmic aaRS enzymes.
aaRS can be divided into two distinct classes based on the structure of the fold of their catalytic domains.
Class I aaRS enzymes are functional monomers that contain a dinucleotide or Rossman fold (RF) of alternating alpha-helices and parallel beta-sheets.
This fold is where ATP and amino-acid binding takes place and therefore facilitates the aminoacylation reaction.
The active site of class I aaRS is marked by the signature motifs ``HIGH'' (His-Ile-Gly-His) and ``KMSKS'' (Lys-Met-Ser-Lys-Ser).
Within the first half of the RF the HIGH motif helps to correctly position the adenine base of ATP and interacts with the phosphates.
The second K of the KMSKS motif is thought to be involved in stabilising the transition state for the primary step of aminoacylation\cite{newberry2002structural}.
Amino acid recognition and binding takes place in the catalytic site when the KMSKS motif is open.
The KMSKS loop closes after the aaRS binds ATP and the aminoacyl-adenylate is formed\cite{kwon2019aminoacyl}.

Class II aaRS enzymes are functional dimers or tetramers with an uncommon catalytic core, comprising seven anti-parallel beta-sheets, flanked by alpha-helices.
Class II aaRS enzymes are defined by three conserved sequence motifs.
Motif 1 is located at the interface of the dimer and enables oligomerization.
Motifs 2 and 3 comprise part of the aminoacylation active site and facilitate amino acid/ ATP binding and adenylate formation.
Motif 3 binds ATP, and motif 2 is involved in coupling ATP and the amino acid and then transferring the amino acid to the 3'-tRNA\cite{kwon2019aminoacyl}.
The distinct active-site structures of class I and II enzymes confer markedly different binding mechanisms for the aminoacylation reaction.
For example, class I aaRSs bind the tRNA acceptor stem via the minor groove side and bind ATP in an extended conformation, whilst class II aaRSs bind the tRNA acceptor stem from the major groove side and bind ATP in a bent conformation.
The two classes of aaRSs split the twenty amino acids into two groups.
Val, Leu, Ile, Met, Glu, Gln, Trp, Tyr, Arg and Cys are activated by their cognate class I aaRS; and Gly, Pro, Ala, Thr, Ser, Hist, Asp, Asm, Lys and Phe are activated by their cognate class II aaRS.
Class I and class II can be further divided into different sub-groups, however that is beyond the scope of this review.
The structural diversity of aaRSs is likely attributable to the need to exclude similar non-cognate amino acids and to discriminate the correct tRNA isoacceptor.

Both class I and class II aaRSs are multi-domain proteins- in addition to their catalytic domains, they include other domains such as their anti-codon recognition domain or an editing domain.
The editing domain found in some aaRSs is to ensure that the essential step of aminoacylation in protein biosynthesis is as accurate as possible, so incorrect amino acids can be removed from aminoacyl-adenylates or mischarged tRNAs\cite{kwon2019aminoacyl}.
Theoretically, it was estimated that mistranslation rate should be approximately 1 in 200 for amino acids differing by just a methyl group (such as valine and isoleucine)\cite{pauling1958festschrift}, however in vivo work demonstrated that the error frequency is closer to 3 in 10,000 (approximately 1 in 3000)\cite{loftfield1972frequency}.
This suggested the existence of proof-reading capabilities of aaRSs, to account for the difference between observed and predicted error rates.
Editing capability has since been shown to be of high functional importance to some aaRSs.
For example, a study in mice in which there was a missense in the editing domain of AlaRS\cite{lee2006editing}.
The impaired proof reading activity of the enzyme lead to an accumulation of misfolded proteins, resulting in the activation of the unfolded protein response and substantial neurodegeneration\cite{lee2006editing}.
Not all aaRSs possess editing activity, only about half do, however the high specificity of the active site of those aaRSs is enough to alleviate proofreading need.

\subsection{Multi-tRNA synthetase complexes}
Higher eukaryotes contain macromolecular complexes, which consist of nine enzymes and three auxiliary proteins, known as multi-tRNA synthetase complexes (MSC).
The 11 cytoplasmic aaRSs not located in the MSC remain free as individual species.
The nine cytoplasmic aaRS enzymes of the MSC are GluRS (EARS1), ProRS (PARS1), IsoRS (IARS1), MetRS (MARS1), GlnRS (QARS1), LysRS (KARS1), ArgRS (RARS1), AspRS (DARS1), LeuRS (LARS1).
GluRS and ProRS are transcribed from a single gene into one polypeptide chain, connected via triple repeats of WHEP domains to form a bifunctional enzyme, EPRS1.
The non-enzyme component of the MSC consists of three aminoacyl-tRNA synthetase-interacting multifunctional proteins (AIMP), AIMP1, AIMP2 and AIMP3.
Human MSCs contain more class II aaRS enzymes than other species, namely DARS1, KARS1, and PARS1, they also contain more auxiliary proteins.
Human MSC components have several additional domains or motifs (Figure \ref{fig:MSC}A), for instance GST-homology domains in EPRS, MetRS, AIMP1 and 2, and WHEP domains in EPRS and MetRS\cite{kim2019evolution, khan20203, kim2020structures}.

The structure of human MSC has not been fully elucidated, however some sub MSC-complex structures have been revealed.
LysRS forms a homodimer and is anchored to the N-terminal peptide region of AIMP2 within the main body of the MSC.
MetRS, AIMP3, EPRS1 and AIMP2 are compactly linked through their GST-homology domains.
ArgRS, GlnRS and AIMP1 assemble into a heterotrimeric complex\cite{kim2019evolution, khan20203, kim2020structures}.
A proposed bisymmetrical model of the human MSC, via homodimerization of AspRS and ProRS, is shown in Figure \ref{fig:MSC}b, based on subcomplex and interaction data\cite{cho2015assembly, kaminska2009dissection, mirande2017aminoacyl}.
This hypothesis proposes that the MSC is a super-complex of two identical, symmetrically arranged subunits (symmetrical along the y-axis in Figure \ref{fig:MSC}b), each containing one copy of the constituent elements, except for LysRS which is present as a dimer in each subunit.

%% Diagram of MSC
\begin{figure}[htb]
\centering
\includegraphics[width=0.6\textwidth]{figures/lit_review/MSC_structure.pdf}
\caption[Human multi-tRNA synthetase complex structure]{The human multi-tRNA synthetase complex (MSC) and its components.
\textbf{A)} The domains of the aminoacyl-tRNA synthetases and auxiliary proteins (AIMP1, 2 and 3) making up the human MSC.
The bifunctional enzyme EPRS1 is made up of the class 1 enzyme GluRS and class II enzyme ProRS (dimer) covalently linked by three WHEP domains.
\textbf{B)} Cartoon representation of a proposed bisymmetrical model structure of the human MSC.
An adaption of a figure created by Myung Hee Kim and Sunghoon Kim\cite{kim2020structures}.
}
\label{fig:MSC}\end{figure}

The function of the MSC was originally thought to be to increase efficiency of protein biosynthesis by localising aaRSs.
Another proposed function of the MSC was to increase stability of its components.
It has been shown using systematic depletion analysis that some of the components are in fact intrinsically less stable in isolation and dependent on their neighbours for stability\cite{han2006hierarchical}.
More recently, examples have emerged where the MSC seems to work as a `molecular reservoir' which can control the release of its components.
The release of components from the MSC has been linked to numerous non-canonical pathways, including cell signalling, metabolism, inflammation and angiogenesis.

Higher eukaryotes usually have extra-domains at the N- or C- terminus of aaRS enzymes compared with lower eukaryotes and prokaryotes, which may partly contribute to MSC assembly.
Most human cytoplasmic aaRS enzymes have at least one new sequence extension or domain, most of which are dispensable for enzymatic activity, suggesting they may contribute to the non-canonical roles of aaRS.
Additionally, aaRSs are often found in the nucleus of cells, where protein biosynthesis does not occur.
The additional evolutionary complexity in human aaRSs and MSC seems to explain the increased physiological complexity and their functionality in non-enzymatic processes.

Examples of non-canonical MSC functionality include--- LARS1 translocating from the MSC to lysosomes, facilitating mTORC1 activation\cite{han2012leucyl}; KARS1 translocating to the nucleus upon immune activation and activating MITF-dependent gene expression in mast cells\cite{yannay2009lysrs}.
Another example is EPRS1 release from the MSC in myeloid cells upon IFN-$\gamma$ stimulation\cite{arif2009two}.
IFN-$\gamma$ induces a network of kinase events (Cdk5, mTORC1 and S6K1 activation) which causes a two-step phosphorylation of two serines in the linker region of human EPRS, and causes its release from the MSC.
EPRS1 combines with other proteins (namely NSAP1, L13a and GAPDH) to form the cytosolic IFN-$\gamma$ activated inhibitor of translation (GAIT) complex, which represses translation of numerous inflammatory-related transcripts, including VEGFA and ceruloplasmin\cite{arif2018gait}.


In addition to the enzymatic components of the MSC, the auxiliary proteins AIMP1, 2 and 3 are also involved in fundamental biological processes.
AIMPs exhibit non-canonical functions aside from their roles as scaffolds in the MSC.
AIMPs have been linked to numerous biological processes, including involvement in immune regulation, nervous system functions, viral replication, genome stability, angiogenesis, and cancer.
AIMP1 interacts with RARS1 and facilitates incoming tRNA substrates to its catalytic site to enhance its enzymatic activity\cite{park1999precursor}.
In addition to improving amino-acyl synthetase activity, secreted AIMP1 has also been shown to be involved in angiogenesis, inflammation induction, wound closure, and maintaining glucose homeostasis\cite{park2006hormonal}.
Transforming growth factor $\beta$ (TGF$\beta$) and the DNA damage response have both been shown to cause phosphorylation of AIMP2 and disassociation from the MSC.
Released AIMP2 has been shown to act as a pro-apoptotic mediator and tumourigenesis suppressor via various pathways\cite{zhou2020roles}.
AIMP3 largely interacts with MARS1, and under conditions initiating the DNA damage response, MARS1 undergoes a conformational change that releases AIMP3 from the MSC\cite{kwon2011dual}.
Released AIMP3 acts as a tumour suppressor, translocating to the nucleus and upregulating expression of the tumour suppressor gene p53.

The functional and structural complexity of the MSC is still being revealed.
The canonical and non-canonical functionality of MSC components promises an unexplored rich source of potential therapeutic targets, but also lends itself to associated pathology.

\section{aaRSs in disease}
Structural and functional variations in aaRSSs' enzymatic and non-enzymatic activities have been linked to various human diseases.
Changes in gene expression, copy number, mutations and genetic variations of aaRSs have been documented in relation to disease\cite{kwon2019aminoacyl}.

Charcot Marie Tooth (CMT) is a genetically and clinically-presenting heterogeneous group of hereditary peripheral neuropathies.
CMT is characterised by progressive degeneration of distal sensory and motor neuron function\cite{yao2013aminoacyl}.
Six aaRSs have been linked to CMT through dominant mono-allelic mutations, including GARS1 and YARS1, which are among numerous genetic-loci to have been linked causally to CMT.
Drosophila models of CMT have demonstrated that CMT-causing YARS1 mutations lead to a conformational change in YARS1, leading to aberrant interactions with transcriptional regulators in the cell nucleus and aberrant expression of certain transcription factors\cite{bervoets2019transcriptional}.

Self-targeting of aaRSs as autoantigens has been implicated in autoimmune diseases.
``Anti-synthetase syndrome'' (ASS) is a heterogeneous group of autoimmune diseases, including interstitial lung disease (ILD), arthritis, idiopathic inflammatory myopathies, myositis and Reynaud's phenomenon\cite{park2008aminoacyl}.
Autoimmune antibodies against histidyl-, threonyl-, alanyl-, isoleucyl-, phenylalanyl-, glycyl-, tyrosyl-, asparaginyl-tRNA synthetase have been found in approximately 30\% of all autoimmune patients\cite{park2008aminoacyl}.
Dysregulation of aaRS has also been noted in other autoimmune diseases, for example multiple sclerosis and immune thrombocytopenia\cite{nie2019roles}.

aaRSs have been linked to viral and bacterial infection.
For example, it has been shown that viral infection leads to the phosphorylation of EPRS and dissociation from the MSC, ultimately blocking PCBP2-mediated mitochondrial antiviral signalling (MAVS) ubiquitination and inhibiting viral replication\cite{lee2016infection}.
Additionally, HIV-1 infection leads to KARS1 release from the MSC, which is partially transported to the nucleus\cite{duchon2017hiv}.
Blocking this release reduced the infectivity of progeny virions, implying that HIV-1 utilizes a dynamic MSC for enhanced viral replication\cite{duchon2017hiv}.
In another study, WARS1 was shown to be increased approximately 27-fold in sepsis patients with a bacterial infection compared with healthy controls\cite{ahn2016secreted}.
Following a range of infections by various pathogens, host monocytes were shown to rapidly secrete WARS1.
The secreted WARS1 increased cell surface levels of CD40, CD80 and CD86, markers of macrophage activation\cite{ahn2016secreted}.

\subsection{aaRSs in cancer}
A growing number of studies have implicated aaRSs and MSC components in tumourigenesis.
Firstly, aaRS enzymatic activity is essential to sustain tumour growth. In cancer metabolism, biosynthesis of aminoacyl-tRNAs has been shown to be highly up-regulated\cite{hu2013heterogeneity}.
In cancer, we see often see dramatic rapid cell growth, this demands an intense increase in overall protein synthesis. To keep up with this demand, the canonical aminoacylation role of aaRSs is crucial as the first step in protein synthesis.

On top of the enzymatic role of aaRSs, their non-canonical functionality has also been associated with both promoting and inhibiting cancer.
The hallmarks of cancer-enhanced growth signalling and proliferation, vascularization, metastasis, altered metabolism, and immune/tumour microenvironment invasion, all have links to aminoacyl-tRNA synthetase function.
Cancer cells require enhanced growth signalling and proliferation to maintain their rapid growth beyond the capacity of normal cells, several aaRSs have been linked to this aberrant growth signalling.
GlyRS has been shown to be integral for cancer-promoting neddylation (where ubiquitin-like protein NEDD8 is conjugated to its target proteins) to occur, and reduced MetRS expression resulted in reduced tumourigenicity in p16INK4a-negative breast cancer cells in vivo\cite{mo2016neddylation, deng2020role, kwon2018stabilization}.
For tumours to grow and metastasize they need to hijack existing vasculature to get blood flow to growing tumour area, or make new vessels by promoting angiogenesis.
Endothelial cells (EC) exposed to TNF-$\alpha$ or VEGF secrete ThrRS.
ThrRS promotes EC migration and angiogenesis.
Inhibition of ThrRS was shown to inhibit angiogenesis, with and without inducing the uncharged tRNA response\cite{williams2013secreted, mirando2015aminoacyl}.
LysRS has been shown to support metastasis by increasing migration.
Following phosphorylation by the MAPK pathway, LysRS binds to the 67kDa membrane bound laminin receptor protein (67LR), preventing its degradation and sustaining laminin-dependent migration.
Once bound to LysRS, 67LR also binds integrin $\alpha$6$\beta$1, which initiates ERK and paxillin signalling, increasing migration by altering cell-cell and cell-ECM adhesion.

aaRSs have also been linked to altering metabolism in cancer.
To make rampant growth feasible, cancer cells adjust metabolism to meet energy demands and provide building blocks for biosynthesis.
LeuRS activates the mTORC1 pathway, which controls translation and autophagy.
Cancer cells utilize the mTORC1 pathway to proliferate more efficiently.
The mTORC1 pathway also causes phosphorylation of EPRS and the release of it from the MSC.
In adipocytes, released EPRS interacts with FATP1 and directs it to the plasma membrane.
Inhibition of FATP1 leads to increased cell viability in breast cancer cell lines, and its expression correlates with decreased patient survival in triple negative breast cancer\cite{mendes2019unraveling}.

\subsection{AIMPs in cancer}
As well as the association between aaRSs and cancer, AIMPs have also been shown to play a role in signalling pathways relevant to numerous cancers.
The MSC-bound aaRSs seem to predominantly promote tumourigenic functions when released from the MSC.
In contrast, the AIMPs bound with them seem to have more tumour-suppressive effects.
AIMP2 has been shown to be a potent tumour suppressor, working via key regulators in the p53, c-Myc, Wnt, TGF-$\beta$ and TNF-$\alpha$ signalling pathways.
Loss of a single allele of AIMP2 in mice resulted in a far higher susceptibility to tumour formation\cite{choi2009multidirectional}.
AIMP1 has also demonstrated tumour-suppressive effects.
In mouse xenograft models, administered AIMP1 was found to reduce tumour volume\cite{han2010antitumor, lee2006antitumor}.
AIMP1 has been shown to induce apoptosis of endothelial cells, such that it supresses tumour vascularization\cite{park2002dose}; it also stimulates anti-tumour immune responses, for example activating NK cells via macrophages, dramatically reducing lung metastasis of melanoma cells\cite{kim2017aminoacyl}.
AIMP3 activates the tumour-suppressor gene p53 following DNA damage or oncogenic stress.
Loss of an AIMP3 allele results in higher susceptibility to spontaneous tumour formation\cite{park2005haploinsufficient}.


\section{aaRSs as therapeutic targets}
aaRSs are considered attractive drug-targets.
Initially, the interest in aaRSs as therapeutic targets arose with the detection of differences between prokaryotic and eukaryotic aaRSs.
Thus, enabling specific targeting of microbial aaRSs with minimal effect on the homologous human aaRSs, making aaRS inhibitors attractive anti-microbial candidates.

\subsection{Antibacterials and antifungals}
In the 1990s, mupirocin (brand name Bactroban) was approved as an antibiotic for the topical treatment of bacterial skin infections.
Mupirocin selectively inhibits bacterial IleRS, by simultaneously occupying isoleucine and AMP binding sites and inhibiting aminoacylation\cite{hurdle2005prospects}.
Mupirocin has shown high selectivity for bacterial IleRS over mammalian IleRS (greater than 8000 fold)\cite{hughes1980interaction}.
This conferred selectivity seems to be due to only a two-amino acid residue difference in the active site of eukaryotic and prokaryotic IleRS\cite{nakama2001structural}.
Another example is Kerydin (Tavaborole or AN2690), an anti-fungal used to treat onychomycosis (a fungal infection of the nail)\cite{fernandes2019boron}.
Kerydin targets the editing site of fungal LeuRS\@.
Kerydin contains boron (Benzoxaborole)\cite{fernandes2019boron}.
The boron atom of Kerydin binds to the terminal tRNA\textsuperscript{leu} ribose, trapping tRNA\textsuperscript{leu} in the editing site, causing a non-productive enzyme conformation and inhibiting protein biosynthesis.

\subsection{Anti-parasitics}
On top of the success of the druggability of aaRS enzymes for bacterial and fungal infections, aaRSs have also shown promise as an anti-parasitic target.
Much like cancer, some parasites are extremely reliant on protein synthesis to keep up with rapid cell growth and continuous proliferation, so are likely to be more sensitive to disruptions to aminoacylation.
Additionally, the evolutionary distance between parasitic aaRSs and human aaRSs is quite large, in fact several parasites have bacterial-like protein translation pathways, not shared by humans\cite{pham2014aminoacyl}.
Numerous aaRSs have shown promise as targets for anti-parasitic agents.
Several naturally occurring compounds target the AsnRS site of parasites, such as \textit{Brugia malayia}, a nematode which causes Lympathic Filariasis.
\textit{Trypanosoma brucei} has also been shown to be susceptible to aaRS inhibition, for example by Benzoxaboroles targeting LeuRS, or by Aminoquinoles and
Benzimidazoles targeting MetRs.
The parasite \textit{Plasmodium falciparum} has been shown to be affected by numerous aaRS inhibitors, including mupirocin, cladosporin and febrifugine derivatives.

\subsection{Febrifugine and its derivatives}\label{subsec:ff}
\textit{Dichroa febrifuga} has been used for centuries in Chinese medicine as an antimalarial remedy, it is considered one of the 50 fundamental herbs.
In 1948, two quinazoline alkaloids, named febrifugine (FF) and isofebrifugine, were first isolated from the plant \textit{Dichroa febrifuga} (Figure \ref{fig:FF_IF_HF})\cite{koepfli1949alkaloids}, as part of a directive to find new anti-malarials from plant sources.
Although febrifugine has excellent anti-parasitic activity, it also has strong liver and gastrointestinal toxicity, limiting its use as a widespread therapeutic.
This motivated the generation of febrifugine derivatives with the hope of reducing off-target toxicity.
The medical applications of the long-used traditional anti-parasitic agent febrifugine and its derivatives have recently attracted much attention.
Febrifugine derivatives have been used to treat malaria, fibrosis, inflammatory diseases and cancer.

%% Structures of FF IF and HF
\begin{figure}[htb]
\centering
\includegraphics[width=0.8\textwidth]{figures/lit_review/compound_sturctures.pdf}
\caption[Prolyl-tRNA synthetase inhibitor chemical structures]{Chemical structures of prolyl-tRNA synthetase (ProRS/PARS1) inhibitors.
Febrifugine and isofebrifugine were first isolated from \textit{Dichroa febrifuga} in 1948.
Halofuginone is a derivative of febrifugine, first synthesized in 1967.
}
\label{fig:FF_IF_HF}\end{figure}

\section{Halofuginone}
One such analogue, a synthetic racemic halogenated derivative of febrifugine, halofuginone (HF; Figure \ref{fig:FF_IF_HF}), was synthesized in 1967 by American Cyanamid Company\cite{zhang2017novel}.
Halofuginone was found to have the most potent anti-malarial activity of the FF analogues in vitro and affected all three stages of \textit{Plasmodium falciparum} (ring stages, trophozoites and schizonts) with equal speed, unlike many other chemicals with antimalarial effects.
The addition of bromine on the quinazoline ring in HF was found not to affect its antimalarial properties, whilst lowering the cytotoxicity for host cells compared to FF.
However, HF does still demonstrate some toxicity to the liver, among other side effects, including diarrhoea and vomiting\cite{pines2015halofuginone}.
In an attempt to reduce the side effects of HF and increase the therapeutic window, trans-enantiomers (2R,3S / +)  and (2S,3R/ - ) of HF have been prepared.
Although (-)-HF was found to have lower toxicity than its optical antipode, it was also found to be less efficacious than (+)-HF\cite{mordechay2021differential, linder20072r}.
This suggests that the biological activity and mammalian toxicity of HF reside with the same enantiomer, therefore there is no advantage to using a specific enantiomer over the racemic mixture.

Recently halofuginone has been researched extensively in association with its applications to non-parasitic diseases.
HF is FDA-approved as a feed additive for poultry to prevent coccidiosis from the protozoa \textit{coccidian}.
HF has also received orphan drug status for scleroderma and Duchenne muscular dystrophy (in which fibrosis is the main complication).
HF has undergone clinical trials as a potential therapeutic in a number of conditions, including cancer\cite{halo2012clin, halo2012clin2}.

\subsection{Halofuginone's antifibrotic properties}
Fibrosis (or fibrotic scarring) is a pathological feature of most chronic inflammatory diseases, which can be induced by a variety of stimuli\cite{wynn2012mechanisms}.
Fibrosis is defined by the accumulation of excess extracellular matrix (ECM) components, especially collagen type I.
If highly progressive, fibrosis can eventually lead to organ malfunction and death\cite{wynn2012mechanisms}.
ECM turnover is altered in most pathological states associated with fibrosis.
TGF$\beta$, the tissue inhibitor of metalloproteinases (TIMPs), and matrix metalloproteinases (MMPs) play an essential role in the regulation of the ECM turnover.
Inflammatory cells are the main source of TGF$\beta$, which induces collagen gene expression and is one of the leading candidates thought to elicit overproduction of ECM proteins in various fibrotic conditions.

Targeting components of the ECM has proved challenging, limiting the success of fibrosis treatment.
HF has been found to have antifibrotic properties and the capability to elicit resolution of established, pre-existing fibrosis, not only act pre-emptively\cite{pines1998halofuginone}.
HF has been shown to reduce collagen synthesis, a hallmark of the disease\cite{pines2001reduction}.
HF is thought to regulate downstream effectors of the TGF$\beta$ signalling pathway by inhibiting Smad3 phosphorylation, which in turn causes a reduction in fibroblast differentiation and levels of ECM proteins\cite{pines2015halofuginone}.

\subsection{Halofuginone and the amino acid starvation response}

Until the last decade, the mechanism of action of halofuginone was unclear, until two papers authored by the same group in 2009\cite{sundrud2009halofuginone} and 2012\cite{keller2012halofuginone} elucidated HF's target and downstream effects.
In the 2009 paper, the group demonstrated using mouse T\textsubscript{H}17 cells that HF activates the amino acid starvation response (AAR) pathway.
Mouse T\textsubscript{H}17 cells were treated with HF or an inactive derivative (MAZ1310) for 3 or 6 hours and microarray analyses were performed.
ATF4 target genes were found to be activated by HF expression, including Asns, Chop, eIF4Ebp, Gpt2, as well as amino acid transport genes, such as Slc6a9 and Slc7a3, both features that correspond with activation of the AAR\@.
Using western blots, the group also showed that general control nonderepressible-2 kinase (GCN2) autophosphorylation was activated by HF treatment, further indicating HF activates the AAR pathway.
This effect was not limited to T\textsubscript{H}17 cells, the AAR pathway was also activated by HF treatment in fibroblasts and epithelial cells\cite{sundrud2009halofuginone}.
However, this paper did not elaborate on how HF activated the AAR\@.

In 2012, the group identified HF's target protein and demonstrated that HF and FF activate the AAR by competing with proline as potent inhibitors of tRNA\textsuperscript{pro} charging activity of EPRS\@.
Rabbit reticulocyte lysate (RRL) was used as an in vitro translation system.
Following supplementation with excess amino acids, only proline was shown to restore translation inhibited by HF in the RRL system.
Moreover, HF-derivatives that were shown to be inactive in functional cell-based assays, such as MAZ1310, also lacked activity in the RRL assay.
Together, this suggests that HF functionality is linked to blocking proline utilization.
To further demonstrate that HF and FF affect proline utilization, the group synthesized DNAs encoding two epitope-tagged polypeptides, one encoding a proline-dipeptide (ProPep), the second encoding a proline-free peptide (NoProPep).
HF and FF treatment prevented translation of ProPep, but had no effect on NoProPep translation\cite{keller2012halofuginone}.

Next, the group investigated the effect of HF on prolyl-tRNA charging and the bifunctional enzyme EPRS (comprised of GluRS and ProRS fused together).
The addition of EPRS from purified-rat-liver reduced the sensitivity of RRL to HF.
They then investigated the inverse using siRNA-mediated knockdown to reduce EPRS levels in lung fibroblasts.
Lung fibroblasts have high levels of EPRS endogenously, so are quite resistant to HF treatment.
The reduction of EPRS levels sensitized the cells to HF treatment and AAR pathway activation--- GCN2 autophosphorylation was induced as well as ATF4 response genes, such as CHOP and ASNS\@.
Together this established for the first time that EPRS is a critical target of inhibition for HF and FF, through which the compounds elicit AAR activation.
The group demonstrated that HF inhibits EPRS in a competitive fashion with proline at the prolyl-tRNA synthetase active site.
HF binding is an ATP-dependent process.
ATP directly locks onto and positions HF onto human ProRS so that one part of HF mimics bound proline and the other mimics the 3' end of bound tRNA\textsuperscript{pro}\cite{zhou2013atp}.
Excess proline addition was shown to abrogate AAR activation and reversed the biological effects of HF\cite{keller2012halofuginone}.

By binding the active site of ProRS, HF blocks proline from binding and inhibits ProRS enzymatic activity.
This results in an intracellular build-up of unaminoacylated (uncharged) tRNA\textsuperscript{pro}s, mimicking the cellular state of proline deficiency, thus triggering the amino acid starvation response.
Uncharged tRNAs bind to the protein kinase GCN2 and stimulates its dimerization and autophosphorylation.
Activated GCN2 phosphorylates eukaryotic translation initiation factor 2A (eIF2$\alpha$), this leads to a reduction in most protein synthesis, whilst increasing translation of ATF4.
ATF4 is a transcription factor of the cAMP response element binding protein (CREB) and induces the expression of many genes involved in the integrated stress response (for example DDIT3/CHOP), amino acid synthetases and transporters, aminoacyl-tRNA synthetases, and autophagy regulators (Figure \ref{fig:HF_AAR})\cite{ye2015gcn2, sundrud2009halofuginone}.

Proline is abundantly incorporated into collagen-- together with hydroxyproline, it constitutes more than 25\% of residues in collagen, which is the predominant protein (80\%) in the ECM\cite{liu2013mirna}\@.
Both hydroxyproline and proline are essential for collagen biosynthesis, structure, and strength\cite{albaugh2017proline}.
So perhaps the anti-fibrotic properties of halofuginone could also be in part because of the proline-richness of collagen, which inhibition of the canonical function of tRNA\textsuperscript{pro} charging interferes with.

%% AAR and HF cartoon
\begin{figure}[htb]
\centering
\includegraphics[width=0.8\textwidth]{figures/lit_review/AAR.png}
\caption[Halofuginone and the amino acid starvation response]{Halofuginone (HF) and the amino acid starvation response (AAR).
HF binds to the catalytic site of prolyl-tRNA synthetase (ProRS) of the bifunctional aminoacyl-tRNA synthetase, EPRS, and causes an accumulation of uncharged tRNAs, mimicking the same cellular environment as if the cell were amino acid deprived.
Uncharged tRNAs bind to the cellular sensor GCN2 and trigger autophosphorylation and activation of GCN2.
Activated GCN2 then phosphorylates eIF2-$\alpha$.
eIF2-$\alpha$-p reduces global protein synthesis, except for mRNAs containing an upstream ORF cluster in their 5' untranslated region (UTR) which are efficiently translated upon eIF2-$\alpha$ phosphorylation\cite{ye2015gcn2}, including the transcription factor ATF4.
Upregulated \textit{ATF4} results in increased expression of many genes involved in stress responses (e.g. \textit{DDIT3} [CHOP]), amino acid metabolism, amino acid synthetases (e.g. \textit{ASNS}) and other aminoacyl-tRNA synthetases.
}
\label{fig:HF_AAR}\end{figure}
%%

%\subsubsection{Halofuginone's downstream signalling pathways}
%--- TO FILL in ---
%To complete...

\subsection{Halofuginone and cancer}\label{sec:HF_cancer}

HF has exhibited anti-cancer effects in numerous studies and different cancers, including metastatic brain tumours, bladder carcinomas, prostate cancer, renal carcinomas, pheochromocytomas, hepatocellular carcinomas, oesophageal squamous carcinomas, lung cancer and breast cancer\cite{abramovitch2004halofuginone, elkin1999inhibition, gavish2002growth, genin2008myofibroblasts, gross2003treatment, nagler2004suppression, wang2020significance, demiroglu2020anticarcinogenic, xia2018halofuginone}.
HF has been shown to exert anti-cancer effects in numerous manners, including reducing tumour growth, reducing angiogenesis, activating autophagy and apoptosis, and disrupting the collagen network of tumours, among other mechanisms.

\subsubsection{Halofuginone and MM}\label{subsec:HF_MM}

As mentioned in Section \ref{sec:MM}, MM is an incurable cancer of plasma cells.
Drug resistance is a severe problem in MM, with patients becoming resistant to previous drug treatments.
Therefore, identifying novel therapeutics for the treatment of MM is of critical importance.

Following the success of HF treatment in numerous preclinical cancer studies and the phase II study of HIV-related Kaposi's sarcoma\cite{koon2011phase}, Leiba et al. (2012) investigated the treatment of HF in multiple myeloma, both in vitro and in vivo\cite{leiba2012halofuginone}.
17 MM cells lines were treated for 48 hours with a range of HF concentrations.
HF was shown to induce a reduction in cell viability in a dose-dependent manner across all 17 MM cell lines, with an IC\textsubscript{50} of approximately 100\si{\nano\Molar} in most cell lines.
The effect of HF on primary cells was then investigated.
BM CD138\textsuperscript{+} cells from five MM patients and PBMCs from two healthy donors were treated with a range of HF concentrations.
A greater dose-dependent reduction in cell viability was seen in the primary MM cells compared to the healthy PBMCs, with an IC\textsubscript{50} ranging from 101-253\si{\nano\Molar} for the MM cells.
Demonstrating that HF specifically inhibits the viability of MM cells while having no significant effect on normal cells (at the given concentration range), this also gave a therapeutic window for HF in MM\@.
Next, the group demonstrated that HF induces apoptosis in MM--- HF treatment triggered caspase 3, 8 and 9 activities in MM cell lines in a dose dependent manner; it increased the quantity of apoptotic cells (Annexin V-FITC apoptosis assay); it caused an accumulation of cells in sub G1 phase of the cell cycle, associated with DNA fragmentation; and it elevated expression of the heat shock protein Hsp-90.
They also showed that exogenous IL-6 and IGF-1, which are central for MM growth and survival, did not rescue HF-induced cytotoxic effects on MM cell lines, indicating that paracrine MM cell growth and the BM environment are unlikely to reverse the biological effects of HF.

The group also exhibited the anti-MM effects of HF in vivo, using in a xenograft model of SCID mice injected with MM.1S cells.
Once tumours reached sufficient size, mice were treated with either PBS or HF for five days a week for the duration of the experiment.
HF treatment was found to inhibit tumour growth and increase overall survival compared to the control mice.

Synergy of HF with existing MM drugs was investigated\cite{leiba2012halofuginone}.
Cells were cultured for 48 hours with HF (25, 50 and 100nM) in combination with 5nM Bortezomib, 25uM lenalidomide, 500nM dexamethasone, or 500nM doxorubicin. Cells were cultured for 24 and 72 hours with HF (25, 50 and 100nM) in combination with 10uM melphalan.
CalcuSyn software (Biosoft, Ferguson, MO, USA) was used to evaluate synergism.
Lenalidomide, dexamethasone and doxorubicin were found to be synergistic or additive with HF in all MM cell lines.
HF showed moderate antagonism in combination with Bortezomib.
However, only one concentration value was used for each of the established MM agents, and a small range of concentrations were used for HF treatment.
A larger range of concentrations of both HF and the other agents would be required to gain a greater insight into the drugs interactions with one another.

From this study, it is clear HF is effective against MM, and could show promise as a potential line of therapy.
However, Leiba et al. (2012) did not show how HF was exerting its effect, or if the AAR was activated.
AAR activation results in upregulated levels of the transcription factor ATF4.
It would be interesting to explore the transcriptional landscape of MM cells and the tumour microenvironment following HF treatment, to see how AAR activation affects this.
The group used MM cell lines, mouse models and primary BM samples from MM patients.
The primary BM samples were compared against healthy donors' PBMCs and not MM patients' own non-myeloma cells, so only limited conclusions can be drawn about HF's specificity for MM cells over normal cells.
Moreover, MM cells are known to interact substantially with their microenvironment.
In this study, the cells were studied in isolation, the effect of HF on the immune microenvironment was not investigated.
The paper states that BM samples were taken from five MM patients, however it is not stated what stage of disease progression the patients were in, or if they were a mixed group of patients in various disease stages.
Therefore, conclusions cannot be drawn whether HF is equally or preferentially effective against newly-diagnosed and relapsed MM patients.

\section{Discussion}

Considering aaRSs are such a highly conserved and ancient family of enzymes, it is surprising how much about their structure and function is still unknown.
Concerted efforts are being made to elucidate the full structure of the MSC, with this knowledge, the full functionality of the MSC might be more clearly understood too.
More non-canonical functions of aaRSs are being revealed, and with it, associated pathologies.
This also presents unexplored potential for aaRS therapeutics.

Drugs targeting aaRSs have shown effectiveness in a wide range of clinical settings including numerous cancers.
aaRS inhibitors are an exciting drug class in the application of cancer, as tumours often have such a high protein biosynthesis burden, especially MM with the overproduction of large quantities of paraprotein.
Therefore, targeting the first step in protein biosynthesis is highly attractive.
Halofuginone, an inhibitor of the ProRS active site of EPRS, has been researched extensively in recent years.
HF has shown anti-MM effects against MM cell lines, in vivo mouse models and primary patient BM samples.

HF has been shown to activate the amino acid response in rabbit reticulocyte lysate and lung fibroblasts.
HF's effects are abrogated by excess proline supplementation.
HF also demonstrates substantial liver and GI toxicity, so may not be able to be used as a widespread drug in cases without orphan drug status.
A ProRS inhibitor which was less toxic, with a wider therapeutic range and whose effects could not be overcome by excess proline would likely be a much more effective anti-cancer agent therapeutically.

The mechanism of action of HF in MM has not been clarified and the transcriptional changes of HF treatment in MM have not yet been described.
MM patient cells have only been studied in isolation-- MM patients' immune microenvironment following HF treatment has not been investigated.
The specificity of HF for MM cells has also not been demonstrated fully.
MM patients' transcriptome, epigenome and genome evolves greatly during disease progression, and the changes are not limited to MM cells and plasma cells.
Therefore, the effect of HF treatment on MM cells must be compared to the patient's own non-myeloma cells to be able to assess specificity, rather than healthy donor's PBMCs, as they are so different from the PBMCs of MM patients.
It would be hugely pertinent to employ single-cell sequencing of BM samples following HF treatment.
scRNA-seq could capture transcriptional changes of the MM cells and their surrounding immune microenvironment, and would also allow composition analysis, so that proportional changes of each cell type could be quantified, and HF's specificity for MM cells evaluated.

So far, only BM samples from MM patients of unknown disease progression have been treated with HF\@.
Therefore, no comment can be made on whether HF works on relapsed patients' MM cells.
Various MM cell lines have been treated with HF, however the resistant variants were resistant to traditional chemotherapy agents (Doxorubicin, Mitoxantrone and Melphalan) or Dexamethasone.
MM is conventionally treated with a three-drug regimen (as discussed in Section \ref{subsec:mm_treatment}), comprising a corticosteroid (e.g. Dexamethasone), a proteasome inhibitor (PI; e.g. Bortezomib or Carfilzomib), and an immunomodulatory drug (IMiD; e.g. Lenalidomide).
MM patients eventually accrue resistance to all three drugs in the regimen.
HF's anti-MM effects were not demonstrated against either PI-resistant or IMiD-resistant cell lines or relapsed patients.
It would be interesting to see if HF's anti-MM effects are maintained in PI-resistant MM cells.
Proteasome inhibition leads to an accumulation of misfolded, damaged or unneeded proteins, which activates the unfolded protein response (UPR), which in part contributes to the anti-MM effects of PIs.
The UPR and AAR share many joint effectors, such as \textit{ATF4} and \textit{DDIT3}, both contributing to ER stress and apoptotic mechanisms.
As there is some overlap with HF's and PIs' mechanisms of action, and that Leiba et al. (2012) reported mild antagonism of HF in combination with the PI bortezomib\cite{leiba2012halofuginone}, you may not expect HF to be effective against PI-resistant MM.
Many new MM agents are approved for relapsed MM initially, rather than as first-line treatments;
therefore, it would be critical for ProRS inhibitors' success in myeloma that their anti-MM effects extend to PI-resistant and relapsed MM.
This is crucial question that must be answered.

Another point that would be of interest to investigate, is the interaction of HF treatment and EPRS's non-canonical functionality.
Following IFN-$\gamma$ stimulation, the EPRS dissociates from the human MSC to participate in the GAIT complex in myeloid cells.
The GAIT complex represses translation of inflammatory-related genes, including \textit{VEGFA}.
It would be interesting to see if HF has any effect on this non-canonical function of EPRS, or in fact, if the GAIT complex impacts HF treatment in myeloid cells.

aaRSs are a very exciting area of research, particularly as drug targets.
So far, the application of aaRS inhibitors in disease has only scratched the surface of their potential as therapeutics.
Much more work is required to fully understand their mechanism of action and breadth of potential applications in disease, particularly MM\@.
